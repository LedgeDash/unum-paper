\section{Background and Motivation}\label{sec:bg}

\begin{table*}[]
\centering
\resizebox{\textwidth}{!}{

	% \begin{tabular}{|l|l|l|l|l|l|l|l|}
	% \hline
	%  & \textbf{Step Functions} & \textbf{Durable Functions} & \textbf{Beldi} & \textbf{Boki} & \textbf{gg} & \textbf{ExCamera} & \textbf{unum} \\ \hline
	% \textit{High-level workflow interface}                        & Y & Y & N & N & Y & Y & Y \\ \hline
	% \textit{Do not need supplemental services/ purely serverless} & N & N & Y & N & N & N & Y \\ \hline
	% \textit{Strong execution guarantee}                           & Y & Y & Y & Y & N & N & Y \\ \hline
	% \end{tabular}

	\begin{tabular}{|l|l|l|l|l|l|l|l|l|l|}
	\hline
	 &
	  \textbf{Step Functions} &
	  \textbf{Durable Functions} &
	  \textbf{Beldi} &
	  \textbf{Boki} &
	  \textbf{Cloudburst} &
	  \textbf{Kappa} &
	  \textbf{gg} &
	  \textbf{ExCamera} &
	  \textbf{unum} \\ \hline
	\textit{High-level workflow interface}                        & Y & Y & N & N & Y & Y & Y & Y & Y \\ \hline
	\textit{Do not need supplemental services/ purely serverless} & N & N & Y & N & N & N & N & N & Y \\ \hline
	\textit{Strong execution guarantee}                           & Y & Y & Y & Y & N & Y & N & N & Y \\ \hline
	\end{tabular}
}
\caption{Comparison with existing work}
\label{table:positioning}
\end{table*}



\begin{itemize}

	\item Benefits of serverless: pay for what you use, event-driven/run only
	when there's useful work to do, highly scalable.

	\item Triggers and Destinations. Dispersed workflow logic, difficult to
	program.

	\item Orchestration-based workflow systems that require adding supplement
	hosted services to the serverless abstraction

\end{itemize}

% In the case of orchestrators/coordinators, functions call back to the
% orchestrator when complete and the orchestrator decides what to do next: which
% downstream function to invoke:

% mu: "mu uses a long-lived coordinator that commands and controls a fleet of
% workers."

% gg: "The main entry-point for executing a thunk is the coordinator program....
% Upon start, this program [the coordinator] materializes the target thunk's
% dependency graph... Then, the thunks that are ready to execute are passed to
% execution engines. ... When the execution of a thunk is done, the program [the
% coordinator] updates the graph by replacing the references to the just-forced
% thunk ... The thunks that become ready to execute are placed on a queue and
% passed to the execution engines when their capacity permits."

% Step Functions: TBA


% unum's different: orchestration logic is distributed to each function. When a
% function complete, instead of calling back to a long-lived coordinator, the
% unum runtime on the Lambda decides what to do next. This design eliminates the
% need for a separate long-running service that executes workflows. Saves a
% communication trip back to the orchestrator.



% ----

% ExCamera or mu:

% all workers in mu use the same generic Lambda function that is capable of
% executing the work of any thread in the computation. (change the programming
% interface: developers not writing individual functions anymore. :"To design a
% computation, a user specifies each worker’s sequence of RPC re- quests and
% responses in the form of a finite-state machine (FSM), which the coordinator
% executes.")

% mu uses a long-lived coordinator that commands and controls a fleet of
% workers. The coordinator steps workers through their tasks by issuing RPC
% requests and processing responses. As examples, the coordinator can instruct
% the worker to re- trieve from or upload to AWS S3; establish connections to
% other workers via a rendezvous server; send data to workers over such
% connections; or run an executable.

% Do the output of one worker go through the coordinator to be sent to and
% consumed by another worker? No, there's a separate rendezvous server for
% worker communication. Like the coordinator, the rendezvous server is long
% lived. mu’s rendezvous is a simple relay server that stores messages from
% workers and forwards them to their destination.


% gg's Restrictions on user code: "gg thunks are designed to be deterministic."
% "it is not allowed to use the network or access unlisted objects or files." So
% functions are basically pure so that gg doesn't need to reason about
% exactly-once semantics.


% "In a long chain
% of rebasing, later threads spend much of their time waiting
% on predecessors (Figure 4). A more-sophisticated launch-
% ing strategy could save money, without compromising
% completion time, by delaying launching these threads."

% [This is another point about many of these frameworks, (I think cloudburst has
% this problem as well) that they need to launch custom-built containers *ahead
% of time*, not on-demand, not event-driven. Double billing becomes an issue.
% Utilization becomes an issue.]

% --- 
% Beldi and Boki.

% --- 
% Cloudburst

% ---
