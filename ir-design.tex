\subsection{\name{} Intermediate Representation}\label{sec:ir}

% \begin{figure*}[t!]
%     \centering
%     \begin{subfigure}[t]{\columnwidth}
%         \centering
%         \begin{minted}[
%             frame=single,
%             fontsize=\scriptsize
%             ]{json}
% {
%     "Name": "F",
%     "Next": {
%         "Name": "G",
%         "InputType": "Scalar",
%         "Conditional": true,
%     }
% }
%         \end{minted}
%         \caption{\texttt{chain} pattern that invokes function \texttt{G} with
%             \texttt{F}'s result}
%         \label{fig:gadget-examples-chain}
%     \end{subfigure}
%     \begin{subfigure}[t]{\columnwidth}
%         \centering
%         \begin{minted}[
%             frame=single,
%             fontsize=\scriptsize
%             ]{json}
% {
%     "Name": "M",
%     "Next": {
%         "Name": "N",
%         "InputType": "Map"
%     }
% }
%         \end{minted}
%         \caption{\texttt{map} pattern that invokes a parallel instance of
%             \texttt{N} for each element of the vector output of \texttt{M}}
%         \label{fig:gadget-examples-map}
%     \end{subfigure}
%     \hfill
%     \begin{subfigure}[t]{\columnwidth}
%         \centering
%         \begin{minted}[
%             frame=single,
%             fontsize=\scriptsize
%             ]{json}
% {
%     "Name": "G",
%     "Next": [
%     {
%         "Name": "H",
%         "InputType": "Scalar"
%     },
%     {
%         "Name": "M",
%         "InputType": "Scalar"
%     }
%     ]
% }
%         \end{minted}
%         \caption{\texttt{fan-out} pattern that invokes function \texttt{H} and
%             \texttt{M} with the result of \texttt{G}}
%         \label{fig:gadget-examples-fanout}
%     \end{subfigure}
%     \begin{subfigure}[t]{\columnwidth}
%         \centering
%         \begin{minted}[
%             frame=single,
%             fontsize=\scriptsize
%             ]{json}
% {
%     "Name": "N",
%     "Next": {
%         "Name": "S",
%         "InputType": {
%             "Fan-in": {
%                 "Values": [
%                 "N-unumIndex-*"
%                 ]
%             }
%         }
%     }
% }
%         \end{minted}
%         \caption{\texttt{fan-in} pattern that invokes function \texttt{S} with
%             the result of all \texttt{N} instances of a \texttt{map}.}
%         \label{fig:gadget-examples-fanin}
%     \end{subfigure}

%     \begin{subfigure}[t]{\columnwidth}
%         \centering
%         \begin{minted}[
%             frame=single,
%             fontsize=\scriptsize
%             ]{json}
% {
%     "Name": "F",
%     "Next": [
%     {
%         "Name": "G",
%         "InputType": "Scalar",
%         "Conditional": "$ret < 0"
%     },
%     {
%         "Name": "H",
%         "InputType": "Scalar",
%         "Conditional": "$ret >= 0"
%     }
%     ]
% }
%         \end{minted}
%         \caption{\texttt{F} branches on the user function's result by
%         combining \texttt{fan-out} with \texttt{Conditional}}
%         \label{fig:gadget-examples-branch}
%     \end{subfigure}

%     \begin{subfigure}[t]{\columnwidth}
%         \centering
%         \begin{minted}[
%             frame=single,
%             fontsize=\scriptsize
%             ]{json}
% {
%     "Name": "F",
%     "Next": {
%         "Name": "F",
%         "InputType": {
%             "Fan-in": {
%                 "Values": [
%                     "F-unumIndex-$0",
%                     "G-unumIndex-($0+1)"
%                 ]
%             }
%         },
%         "Conditional": "$0 < $size-1"
%     },
%     "Next Payload Modifiers": ["$0=$0+1"]
% }
%         \end{minted}
%         \caption{\texttt{fold} pattern that folds \texttt{F} on the output of
%         a vector of \texttt{G} functions}
%         \label{fig:gadget-examples-fold}
%     \end{subfigure}

%     \caption{The \name{} IR representation of common transition patterns.}
%     \label{fig:ir-examples}
% \end{figure*}


\begin{figure}[t!]
    \centering
    \begin{minted}[
        frame=single,
        fontsize=\scriptsize
        ]{yaml}
Name: "Name of the source function"
Next: "One or a list of transitions"
    Name: "Name of the target function of this transition"
    InputType: "Input type of this target function
               Support the three different types below"
        Scalar: "Target takes only the source's output
                as an entire entity"
        Map:    "Treat source's output as an iterable.
                Send each element to an instance of
                the target function"
        Fan-in: "Target function take sources and other
                functions' outputs as input. Run target
                only when all functions in Values complete"
            Values: "A list of function instances whose output
                    the target function depends on as input"
    Conditional: "Boolean expression. Run this
                 transition only when true."

    \end{minted}
    \caption{Primitives for encoding the outgoing edges transitions from a
    source function to one or more target functions.}
    \label{fig:ir-schema}
\end{figure}



\begin{figure}[t!]
    \centering
    \begin{subfigure}[t]{\columnwidth}
        \centering
        \begin{minted}[
            frame=single,
            fontsize=\scriptsize
            ]{yaml}
Name: "N"
Next:
    Name: "S"
    InputType: 
        Fan-in:
            Values: ["N-unumIndex-*"]
        \end{minted}
        \caption{fan-in pattern that invokes function \texttt{S} with the
            result of all \texttt{N} instances of a \texttt{map}.}
        \label{fig:gadget-examples-fanin}
    \end{subfigure}

    \begin{subfigure}[t]{\columnwidth}
        \centering
        \begin{minted}[
            frame=single,
            fontsize=\scriptsize
            ]{yaml}
Name: "F"
Next:
    Name: "F"
    InputType:
        Fan-in:
            Values: ["F-unumIndex-$0","G-unumIndex-($0+1)"]
    Conditional: "$0 < $size-1"
Next Payload Modifiers: ["$0=$0+1"]
        \end{minted}
        \caption{\texttt{fold} pattern that folds \texttt{F} on the output of
        a vector of \texttt{G} functions}
        \label{fig:gadget-examples-fold}
    \end{subfigure}

    \caption{The \name{} IR example of fan-in and fold.}
    \label{fig:ir-examples}
\end{figure}

Before executing serverless workflows in a decentralized manner, a workflow
definition first needs to be partitioned such that each function performs only
a subsection of the workflow. The system needs to decide where to make the
partition, how big each parition is and how to assign partitions to user
functions. \shadi{this whole description "partition" seems like adding an unnecessary new term and detail. can we just use the terms used so far? also, can we just refer back to patterns?}

\name{} partitions a workflow by converting it into an intermediate
representation. From a high-level workflow definition, \name{} first derives a
directed graph representation where nodes are user functions and edges are
workflow transitions between functions. Then each function is assigned to
perform just the transitions of its node's outgoing edges. \shadi{can we maybe call this "unum config" to be consistent with architecture?}

Compared with workflow definitions where developers compose functions with
higher-level building blocks such as patterns, the \name{} IR encodes workflow
transitions with low-level primitives. In particular, the IR is made up of a
set of configurations, one for each function in the workflow, that just
specifies i. \textit{which} function(s) to run next, ii. \textit{when} the
next function(s) should run, iii. \textit{what} input should the next
function(s) run with. Figure~\ref{fig:ir-schema} shows the main primitives in
the \name{} IR language for expressing workflow transitions.

With these primitives, the IR can express all common workflow patterns
(\S\ref{sec:transition-patterns}) using just a few lines of code for simple
patterns and a combination of the primitives for more complex interactions.
For instances, a chain from function \texttt{F} to \texttt{G} can be expressed
simply as:
\begin{minted}[
            fontsize=\scriptsize
            ]{yaml}
Name: "F"
Next:
    Name: "G"
    InputType: "Scalar"
\end{minted}

And Figure~\ref{fig:ir-examples} gives an example of a regular fan-in and
using a combination of \texttt{Fan-in}, \texttt{Conditional} and \texttt{Next
Payload Modifier} to express fold.

The IR configuration is platform-agnostic. In particular, it does not specify
how to launch the next function(s), and how to send input data to the next
function(s). These details are left to the \name{} runtime library that can
target specific platforms (for example, the runtime for AWS runs functions as
lambdas and uses the Lambda asychronous invocation API to launch functions).
Thus, workflows are portable across platforms by simply changing the runtime
library.

\subsubsection{Naming and Dynamic Runtime Behavior}\label{sec:ir:naming}

A key design challenge of the \name{} IR is to support dynamic runtime
behavior with statically generated configurations. Importantly, there might be
multiple \emph{runtime} instances of the same function during execution.
\name{} needs to uniquely identify each function instance to correctly perform
transitions. For example, in a workflow that consists of a map followed by a
fan-in, map creates multiple parallel instances of the same function, but the
number of instances depends on the output data and cannot be known at
compile-time. \name{} needs to identify each function instance so that fan-in
can execute correctly (e.g., not miss an instance or counting the same
instance twice).

To solve this problem, \name{} defines a naming scheme for runtime instances
and exposes a set of APIs to interact with names. First,
\name{} requires each user function to have a user-defined name (in the
top-level \texttt{Name} field). This is also a common requirement for existing
serverless systems when developers deploy their functions. Next, each branch
in a map and fan-out pattern is assigned an integer index,
starting from zero, and the $i^th$ branch is named
\texttt{<FunctionName>-unumIndex-\emph{i}}. For nested fan-outs, the indexes
are delimited with periods. For example,
\texttt{<FunctionName>-unumIndex-\emph{i.j}} identifies the $i^{th}$ branch in
the outer loop and $j^{th}$ in the inner loop.

To identify all branches, the \name{}~IR supports glob patterns, such as
\texttt{*}, when specifying runtime instances' names.
Figure~\ref{fig:gadget-examples-fanin} shows a fan-in example that invokes
\texttt{S} with the outputs from all instances of \texttt{N}.

\name{} IR provides a set of runtime variables and APIs that enables
additional dynamic behavior. For example, \texttt{\$0} refers to the
inner-most index of the function instance which can be used in the
\texttt{Values} field for fan-in when input depends on specific instances
instead of all instances. The \texttt{Next Payload Modifiers} field allows
expressions that can modify the runtime variables (e.g., \texttt{\$0=\$0+1}),
which can implement fold or for loops~\ref{fig:gadget-examples-fold} when
combined with a termination condition in \texttt{Conditional}.

% The \name{} frontend compiler can generate IR from workflow definitions
% written in high-level description languages (e.g., AWS Step Functions), but
% developers can also write IR configurations directly, with just a few lines of
% JSON. In particular, The \name{} IR uses a top-level \texttt{Name} field to
% identify the source function that the configuration is co-located with, a
% \texttt{Next} field to list the workflow transitions, and under which a
% \texttt{Name}, \texttt{Conditional}, and \texttt{InputType} fields that
% identify which function to run, when and with what input.
% Figure~\ref{fig:ir-examples} shows examples on how to express common
% orchestration patterns such as branching, chaining, map, fan-out, fan-in and
% fold: 


% \paragraph{chain}

% %Chaining is a simple but common orchestration pattern. For example, an
% %application might include a function (the source) that processes input data
% %from a sensor followed by another function (the target) that adjusts an
% %actuator based on the processed input. The \texttt{chain} pattern connects the
% %two functions by invoking the target function with the source function's
% %result.

% Figure~\ref{fig:gadget-examples-chain} expresses a chain with a \texttt{Next}
% field that contains a single JSON object whose \texttt{InputType} is
% \texttt{Scalar}. \texttt{Scalar} means that the target function (i.e.,
% \texttt{G}) should receive as input the source function's (i.e., \texttt{F}'s)
% output as a singular entity. The \texttt{Conditional: true} means that this
% transition should always take place whenever \texttt{F} produces a result.
% \texttt{Conditional} is assumed to be \texttt{true} when not specified.

% \paragraph{fan-out}

% %Another common pattern processes the output of a function in different ways in
% %parallel. For example, an social network application might perform several
% %independent functions given a new user post, such as URL shortening and
% %resolving other users mentioned in the post. The \texttt{fan-out} pattern
% %launches a vector of functions (branches) each with the output of the same
% %source function.

% Figure~\ref{fig:gadget-examples-fanout} expresses a fan-out with an array in
% the \texttt{Next} field. Each item of the array identifies a different target
% function that should should run in parallel. The \texttt{InputType} field is
% \texttt{Scalar} for both branches because both should receive as input the
% source function's (i.e., \texttt{F}'s) output as a singular entity.

% \paragraph{map}

% %An application may also perform the same function on multiple outputs of a
% %source function. For example, a photo management application might unpack an
% %archive of high-resolution images in one function and perform compression on
% %each of the resulting images. The \texttt{map} pattern invokes the same
% %function once for each element in a vector of outputs from the source
% %function.

% Figure~\ref{fig:gadget-examples-map} expresses a \texttt{map} with a
% \texttt{Next} field that contains a single JSON object whose
% \texttt{InputType} is \texttt{Map}, which indicates that the source function's
% (i.e., \texttt{F}'s) output should be treated as an array and for each
% element, a target function (i.e., \texttt{G}) should launch with that element
% as the input.

% \paragraph{fan-in}

% %After processing data with many parallel branches, applications commonly want
% %to aggregate results. For example, a video encoder might divide a large video
% %into chunks, encode each in parallel and then concatenant all the encoded
% %chunks together. The \texttt{fan-in} pattern invokes a single ``sink''
% %function with the outputs from a vector of functions (the fan-in branches).

% Figure~\ref{fig:gadget-examples-fanin} expresses a \texttt{fan-in} with a
% \texttt{Next} field that contains a single JSON object whose
% \texttt{InputType} is \texttt{Fan-in} under which a \texttt{Values} fields
% specifies an ordered list with names of functions whose results should be the
% target function's (i.e., \texttt{S}'s)input.

% Different from \texttt{Scalar} and \texttt{Map} where the target function only
% depends on the source function's output, \texttt{Fan-in} expresses data
% dependencies across functions which is difficult to implement efficiently. We
% discuss how we support \texttt{fan-in} in \S\ref{sec:runtime}.

% \paragraph{branch}

% %An application may decide to take a different branch of the workflow graph
% %based on runtime behavior (e.g., the output of a function). The
% %\texttt{branch} pattern selects a next function based on some boolean
% %condition.

% Figure~\ref{fig:gadget-examples-branch} expresses a \texttt{branch} with an
% array in the \texttt{Next} field where each branch has a \texttt{Conditional}.
% Notice that \texttt{branch} is expressed similarly as \texttt{fan-out}
% combined with the \texttt{Conditional} field. This example demonstrates the
% flexibility and expressiveness of \name{} IR's building blocks which
% developers can compose to express additional patterns.

% \paragraph{fold}

% In functional programming, \texttt{fold} is higher-level function that takes
% another functions and applies it to an vector of values. In serverless
% workflow, \texttt{fold} is an advanced pattern that is supported by few
% systems~\cite{azure-functions}. Although \name{} is built on a simpler
% abstraction without adding new components, we support folding a function on
% the output of a vector of functions.

% Figure~\ref{fig:gadget-examples-fold} expresses a \texttt{fold} that folds the
% function \texttt{F} on a vector of \texttt{G} instances result, using a
% combination of \texttt{Fan-in}, \texttt{Conditional} and \texttt{Next Payload
% Modifiers}. On a high-level, \texttt{F} recursively calls itself, passing the
% previous iteration's output (\texttt{F-unumIndex-\$0}) and the next \texttt{G}
% instance's result (\texttt{F-unumIndex-(\$0+1)}), until it reaches the last
% \texttt{G} instance in the vector ((\texttt{Conditional: "\$0<\$size-1"})). On
% each iteration, \texttt{F} increments its index by 1 (\texttt{"\$0=\$0+1"}).
% We explain the details on \name{} IR's naming scheme and the use of index and
% runtime variables (e.g., \texttt{\$0}) in the next section.

