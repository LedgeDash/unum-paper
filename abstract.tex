Serverless workflow orchestrators make it easier for developers to build complex
serverless applications by driving the execution of a workflow from a central
service.  Making orchestrators fault-tolerant and scalable requires engineering
effort as well as large upfront resource costs to deploy. As a result,
developers typically rely on multi-tenant services run by cloud providers that
can amortize these costs across many users.

Unfortunately, this centralization comes at the expense of application
flexibility. Orchestrators support a fixed set of interactions, and trade-off
performance, scalability, execution guarantees, and resource efficiency using a
policy chosen by the cloud provider. They support many applications well, but
cannot simultaneously satisfy the needs of \emph{all} applications.

We argue that centralized serverless orchestrators are actually unnecessary.
Instead, decentralized \emph{library} orchestrators can support the same
application patterns and execution guarantees as existing orchestrators using
only the basic building blocks of serverless platforms---a scalable ``Function''
executor (FaaS) and a consistent scalable datastore. This enables different
applications to use or customize different libraries to implemented different
patterns or make application-specific trade-offs.

To demonstrate this, we describe the design and implementation of \name{}, a
decentralized serverless workflow system. \name{} compiles workflows written in
existing high-level workflow languages a partitioned orchestration library that
executes orchestration logic in-situ with user functions. Unum provides the same
strong execution guarantees as centralized orchestrator, performs similarly,
scales better, and costs less than existing centralized workflow systems.

Our implementation of \name{} supports a superset of applications the AWS Step
Functions orchestrator supports, runs on both AWS Lambda and Google Cloud
Functions, and runs as much as 2x faster and up to an order of magnitude cheaper
than native Step Functions on a representative set of applications.
