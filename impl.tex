\section{Implementation}\label{sec:impl}

We implement a prototype \name{} runtime that supports AWS and Google Cloud.
We also implement a front-end compiler that transforms AWS Step Function
definitions to \name{} IR. Currently our runtime only supports Python
functions and is itself written in 1,119 lines of code. The Step Functions
compiler is 549 lines-of-code.

Implementing the runtime primarily requires specializing high-level
functionality the IR depends on for a particular FaaS platform and data store.
The FaaS platform must support asynchronous invocation and the data store must
be strongly consistent with support for atomic creation and set operations.

Importantly, we choose data stores and primitives that only incur per-use
costs and scale on-demand. For example, we use DynamoDB in on-demand capacity
mode, rather than provisioned capacity mode, and avoid long-running services
such as a hosted Redis or cache. As a result, \name{} incurs fine-grained
costs only when performing orchestration (e.g., per-millisecond Lambda runtime
costs to execute the \name{} library, per-write DynamoDB costs to create
checkpoints).

\subsection{AWS Lambda \& DynamoDB}

Asynchronous invocation in Lambda is natively supported. In particular, the
Lambda \texttt{Invoke} API is asynchronous when passed
\texttt{InvocationType=Event}. In the event of a crash, we use Lambda's
Failure Destination~\cite{aws-lambda-failure-destination} to redirect the
fault to an error handler function which runs just the \name{} runtime. The
error handler checks if the failed function should be retried (e.g., based on
the Step Function definition~\cite{aws-step-functions-retry}) and if so,
retries the function by explicitly invoking it again.

DynamoDB organizes data into tables, with each item in a table named by a key.
Within tables, items are unstructured by default. Our implementation of
\name{} uses a single table for each workflow. Each item in the table
corresponds to a checkpoint, or a coordination set for fan-in or garbage
collection.

DynamoDB supports atomic item creation by passing the conditional flag
\texttt{attribute\_not\_exists} to the \texttt{put\_item} API call. We use
this for creating both checkpoint blobs and coordination sets. DynamoDB
supports set addition natively using the \texttt{Map} field type. In
particular, we use update expressions to atomically set a named map element to
true.  As an optimization, we use the \texttt{ALL\_NEW} flag when adding to a
set to atomically get the new value after a set in a single operation.

\subsection{Google Cloud Functions \& Firestore}

Google Cloud Functions (GCF) do not have an asynchronous invocation API.
Instead, we allocate function-specific pub-sub queues and subscribe each
function to its respective queue.  \name{} then asynchronously invoke a
function by publishing the input data as an event to the function's queue.

GCF supports automatic retry for asynchronous
functions~\cite{google-cloud-functions-retry}. In the event of a crash, the
\name{} runtime in the retry execution checks if the failed function should be
retried and if so, retries the function by explicitly invoking it again.

Firestore organizes data into logical collections (which are created and
deleted implicitly) containing unstructured items, named by a unique key.
Similar to DynamoDB, we use a separate collection for each workflow. Atomic
item creation is supported using a special \texttt{create} API call, which
only succeeds if the key does not already exist. Firestore supports an
\texttt{Array} field type which can act as a set by using the
\texttt{ArrayUnion} and \texttt{update} operation, which atomically sets the
field to the union of its existing elements and the provided elements. The
\texttt{update} operation always returns the new value data.
