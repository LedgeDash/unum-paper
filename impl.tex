\section{Implementation}\label{sec:impl}

We implement a prototype of \name{} that runs Python functions on AWS Lambda
with DynamoDB or S3 as the intermediary data store. The frontend compiler
supports workflows written as AWS Step Functions state machines. The compilers
are written in Python with 1,277 lines of code. The runtime are written in
Python with 2,957 lines of code. We also integrate our toolchain with AWS SAM
to build and deploy \name{} workflows as Amazon CloudFormation stacks, which
automates the development process. Building and deploying an \name{} workflow
takes only one simple command.

The runtime uses the \texttt{boto3} library to interact with AWS services.
Lambda functions are invoked asynchronously (\texttt{InvocationType='Event'})
with the \texttt{invoke()} API. Checkpointing to DynamoDB is a
\texttt{PutItem} operation with a condition that the item does not already
exist (\texttt{ConditionExpression='attribute_not_exists(Name)'}). Check
checkpoint existence (\texttt{get_checkpoint()}) is a consistent read via
\texttt{GetItem} operation. For fan-in functions, reading input from DynamoDB
uses the \texttt{BatchGetItem} operation with consistent reads.

\name{} by default uses the on-demand capacity mode when creating DynamoDB
tables where applications are charged by the amount of read and write it
performs instead of a fixed hourly rate.

The runtime for S3 uses \texttt{}