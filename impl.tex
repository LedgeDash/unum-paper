\section{Implementation}\label{sec:impl}

We implement a prototype \name{} runtime that supports AWS Lambda and Google
Cloud Functions, leveraging DynamoDB and Firestore, respectively, as
intermediate datastores. We also implemented a front-end compiler that takes
arbitrary AWS Step Function state-machine definitions and compiles them to
\name{} IR, which can be run on either. Currently our runtime only supports
Python functions and is itself written in 1,119 lines-of-code. The Step
Functions compiler is 549 lines-of-code.

Implementing the runtime primarily requires specializing high-level
functionality the IR depends on for a particular FaaS platform and datastore.
The FaaS platform must support asynchronous function invocation and the
datastore must be linearizable with support for atomic creation and set
operations.

Importantly, we choose datastores and primitives that only incur per-use costs,
rather than including baseline provisioning cost, and scale on-demand. For
example, we use DynamoDB in on-demand capacity mode, rather than provisioned
capacity mode, and avoid long-running services such as a hosted Redis or cache.
As a result, \name{} workflows, like their centralized counterparts, incur
resource costs \emph{only} to actually execute a workflow.

\subsection{AWS Lambda \& DynamoDB}

Asynchronous invocation in Lambda is natively supported. In particular, the
Lambda \texttt{Invoke} API is asynchronous when passed.

DynamoDB organizes data into tables, with each item in a table named by a key.
Within tables, items are unstructured by default. Our implementation of \name{}
uses a single table for each workflow invocation. Each item in the table
corresponds to a checkpoint, or a coordination set for fan-in or garbage
collection.

DynamoDB supports atomic item creation by passing the conditional flag
\texttt{attribute\_not\_exists} to the \texttt{put\_item} API call. We use this
for creating both checkpoint blobs and coordination sets. DynamoDB supports set
addition natively using the \texttt{Map} field type. In particular, we use
update expressions to atomically set a named map element to true.  As an
optimization, we use the \texttt{ALL\_NEW} flag when adding to a set to
atomically get the new value after a set in a single operation.

\subsection{Google Cloud Functions \& Firestore}

Google Cloud Functions (GCF) do not have an asynchronous invocation API.
Instead, we allocate function-specific publish-subscribe queues and subscribe
each function to its respective queue.  \name{} then performs asynchronous
invocation of a function by publishing the input data as an event to the
function's queue.

Firestore organizes data into logical collections (which are created and deleted
implicitly) containing unstructured items, named by a unique key. Similar to
DynamoDB, we use a separate collection for each workflow. Atomic item creation
is supported using a special \texttt{create} API call, which only succeeds if
the key does not already exist. Firestore supports an \texttt{Array} field type
which can act as a set by using the \texttt{ArrayUnion} and \texttt{update}
operation, which atomically sets the field to the union of its existing elements
and the provided elements. The \texttt{update} operation always returns the
new value data.
