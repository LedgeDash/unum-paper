\section{Design}\label{sec:design}

\name{} orchestrates the execution of workflows a decentralized manner on top of
a FaaS platform and a scalable consistent storage service. Workflows are modeled
as a directed execution graph where nodes represent user-defined FaaS functions
and each edge represents an invocation of one functions (incoming edge) with the
output of one or more other functions (outgoing edges). The FaaS platform
executes each function invocation at least once and may execute it more than
once due to lost messages, infrastructure failure, inconsistency in invocation
queues, etc.

An \name{} graph may can include fan-outs, where output from a node is used to
invoke several functions or split up and ``mapped'' multiple times on the same
function. Each such branch may be taken conditionally, based on the output value
or dynamic states of the graph. Execution graphs may also contain fan-ins, where
the outputs of multiple nodes are used to invoke a single aggregate function.
Cycles are also supported and each iteration through a cycle is a different
invocation of the target function.

\begin{table}[]
  \centering
  \begin{tabular}{|m{0.2\linewidth}|m{0.75\linewidth}|}
    \hline
	\textbf{Workflow} & A directed graph of functions that takes an input and produces one or more outputs. \\
    \hline
	\textbf{Function} & A user-defined FaaS function, linked with the \name{} runtime library. \\
    \hline
	\textbf{Invocation} & An invocation of a function one request of the FaaS platform to run a function and is associated with an argument to the function. \\
    \hline
	\textbf{Execution} & The FaaS platform may attempt to \emph{execute} a function invocation one or more times, and guarantees that at least one execution completes. \\
    \hline
  \end{tabular}
  \caption{\name{} terminology.}
  \label{table:terms}
\end{table}

\subsection{Architecture}\label{sec:design:architecture}

\begin{figure*}[t!]
	\centering
	\begin{subfigure}[t]{0.8\textwidth}
	\centering
		\includegraphics[width=0.8\columnwidth]{figures/unum-arch-compile-time.pdf}
		% \includegraphics[width=\columnwidth]{figures/architecture.png}
		\caption{Serverless workflows form directed graphs. \name{}
		partitions the graph into an intermediate representation where each
		function is embedded with an \name{} configuration that encodes how to
		transition to its immediate downstream nodes. Developers package user
		function, \name{} config and \name{}'s runtime library (a pair of
		ingress and egress components) together to create unumized functions.}
		\label{fig:arch:unum-compile-time}

	\end{subfigure}
	\begin{subfigure}[b]{\columnwidth}
		\centering
		\includegraphics[width=0.8\columnwidth]{figures/unum-arch-centralized.pdf}
		\caption{A typical serverless workflow system drives workflow logic
			using a centralized orchestrator that invokes constituent
			functions and waits for their outputs.}
		\label{fig:arch:centralized}
	\end{subfigure}
	\hfill
	\begin{subfigure}[b]{\columnwidth}
		\centering
		\includegraphics[width=.7\columnwidth]{figures/unum-arch-runtime.pdf}
		\caption{At runtime, \name{} orchestration logic is decentralized and
			runs in-situ with the user functions on an unmodified serverless
			platform. For synchronization and checkpointing,
			\name{} relies exclusively on a standard datastore of choice, such
			as DynamoDB or Cosmos DB.}
		\label{fig:arch:unum-runtime}
	\end{subfigure}
	\caption{\name{}'s Decentralized Orchestration. \name{} partitions
	orchestration logic at compile time and a \name{} runtime runs in-situ
	with user functions to perform only the orchestration logic local to its
	subsection of the graph.}
	\label{fig:arch}
\end{figure*}

Figure~\ref{fig:arch:unum-compile-time} depicts how a developer goes from a high
level workflow description and functions to running the workflow in a
decentralized manner using \name{}.

Developers write individual functions and describe the workflow using a
high-level workflow language, such as Step Function's expression language. They
pass this description and functions through a front-end \name{} compiler that
extracts portable \name{} IR for each node in the graph and ``attaches'' it to
the function (e.g.\ by placing a file containing the IR alongside the function
code). A backend \name{} linker ``links'' each function with a
platform-specific \name{} runtime library.~\footnote{Since functions are
typically written in dynamic languages, the \name{} library source code is
placed alongside the function and dynamically imported, rather than statically
linking an object file}

Runtime libraries are specific to the FaaS platform and datastore---e.g.\ Amazon
Lambda and DynamoDB, or Google Cloud Functions and Firestore. Each runtime is
composed of an ingress and egress component that run, respectively, before and
after the user-defined function. The egress component coalesces input data from
each incoming edge (e.g.\ in a fan-in), resolves input data if passed by name
rather than by value, and passes the input value to the function. The egress
component uses the function's result to invoke the next function(s) as
determined by the node's IR, ensures execution semantics using checkpoints,
performs coordination with sibling branches in fan-in, and deletes intermediate
states no longer needed for the workflow.

Developers deploy each linked function along with its IR to the FaaS platform.
The workflow is invoked by invoking the first function in the graph.  The
\name{} runtime is responsible for interpreting the \name{} IR at each node and
invoking next functions, performing coordination, checkpointing, and garbage
collection to execute the workflow in-situ with functions, in lieu of a
centralized orchestrator (Figure~\ref{fig:arch:unum-runtime}).

\subsection{\name{} Intermediate Representation}\label{sec:design:ir}


\begin{figure}[t!]
    \centering
    \begin{minted}[
        frame=single,
        fontsize=\scriptsize
        ]{rust}
type NodeName String
type InvocationName
    (SessionId, NodeName, Iteration, Vec<FanOut>)

struct IR {
  name: NodeName,
  next: Vec<Conditional<Invoke>>
}

enum Invoke {
  Scalar(NodeName),
  Map(NodeName),
  FanIn(NodeName, Vec<InvocationName>)
}

type Conditional<T> Fn(Input, Output) -> Option<T>
    \end{minted}
    \caption{\name{} IR}
    \label{fig:irschema}
\end{figure}

\section{Design}

To achieve the objectives in \S\ref{sec:goals}, \name{} utilizes a strategy we
call ``\emph{decentralized orchestration}'' where instead of executing
workflow orchestration logic entirely within a centralized orchestrator, a set
of ``decentralized orchestrators'' run \emph{in-situ} with user functions and
each performs only the orchestration logic \emph{local to its subsection} of
the workflow.

Efficiently implementing decentralized orchestration while also preserving the
benefits of centralized orchestrators (\S\ref{sec:bg:orchestrator}) requires
\name{} to solve three key challenges:

\squishenum
	\item How to partition its orchestration logic such that it can run in a
	decentralized manner in-situ with user functions given a high-level workflow description?

	\item How to efficiently execute orchestration logic in a
	decentralized manner, esp. when it requires data sharing and
	synchronization across function instances (e.g., fan-in)?

	\item How to provide exactly-once semantics when the
	orchestration logic is decentralized across function instances that may
	crash or retry mid-execution?
\squishenumend

\name{} contributes an end-to-end system that solves the three challenges and
delivers significant cost savings and improved or comparable performance than
a state-of-the-art production orchestrator (\S\ref{sec:eval}).
Figure~\ref{fig:arch} depicts \name{}'s architecture. \shadi{we should reference a,b,c individually as some later parts in the text.  btw, isn't 1.a the architecture rather than 1?}

\name{} solves the first challenge \emph{at compile time}, using a frontend
compiler and an intermediate representation (IR). Given a workflow definition
written in a high-level description language, the frontend compiler derives a
directed graph representation where nodes are user functions and edges
represent workflow transitions between functions. Based on the directed graph,
the compiler generates an IR in the form of configuration files, one file for
each node in the graph. A \textit{\name{} configuration} encodes all the outgoing edges
of a node such that each function in the workflow knows how to transition just
to its immediate downstream node(s).

\name{} solves the second challenge using an \textit{\name{} runtime} library
that efficiently implements a set of workflow patterns in a decentralized manner
and can run in-situ with user functions. In particular, the \name{} runtime is
made up of an ingress component and an egress component.  Developers package
each user function with its assigned \name{} configuration and the \name{}
runtime to create an \emph{unumized} function. When an unumized function
executes, its entry point is no longer the user code but instead the \name{}
ingress. And when user code completes, it returns its results to the \name{}
egress which then interprets its co-located \name{} configuration and performs
the workflow transition.

A critical complexity is to enable data sharing and synchronization across
functions in a workflow invocation. For example, applications might aggregate
the results of multiple upstream functions with a single ``sink'' function
when all of the upstream functions complete (commonly called a ``fan-in''). To
this end, \name{} leverages a shared \emph{intermediary data store} that is a
strongly consistent serverless storage (e.g., DynamoDB). Upstream functions
each write their output into the intermediary data store and use this data
store to share data and  synchronize across each other.


\name{} also leverages the intermediary data store to solve the third
challenge and provide a strong exactly-once semantics. \name{} uses a
checkpointing mechansim to limit the scope of retries when workflows crash
mid-execution and guarantee that even if there are multiple instances of the
same function, concurrent or not, only one instance's output is taken as the
final result and propargates downstream. When user functions do not produce
externally visible side-effects, \name{}'s exactly-once guarantee appears the
same as exactly-once semantics.

In the rest of the section, we first highlight a set of common workflow
patterns that \name{} supports (\S~\ref{sec:transition-patterns}), then
describe how the \name{} runtime efficiently executes workflows in a
decentralized manner (\S~\ref{sec:runtime}). Next, we show how the \name{} IR
encodes transitions between functions (\S\ref{sec:ir}) and allows partitioning
of workflows written as a high-level description. Finally, we detail \name{}'s
checkpointing mechanism and how it ensures strong execution guarantee
(\S~\ref{sec:exec-gntee}).


\subsection{\name{'s} transition patterns}
\label{sec:transition-patterns}


We have identified six workflow transition patterns as the main building blocks to build arbitrary workflow compositions. These patterns are simple to use and yet expressive. We found these to be sufficient to express all commonly used orchestrator systems including AWS Step
Functions, \shadi{X, y, and Z} and all serverless applications we encountered \shadi{cites}. \shadi{should we add sentence on how fold and for are unique?}
\begin{itemize}
	\item \textit{\textbf{chain:} }
		Chaining is a simple but most fundamental control-flow pattern, e.g.,  a function processing sensor data followed by another function adjusting an	actuator. The \texttt{chain} pattern connects the
		two functions by invoking the target function with the source function's
		result. 

	\item \textit{\textbf{fan out:}} 
	A common pattern to create parallel processing,  e.g.,  an social network application processing a new user post by, in parallel, performing
	 URL shortening
	and resolving users tagged. 
	The fan-out pattern
	launches a vector of functions (branches) with the output of a single
	source function. 
	\item \textit{\textbf{map: }} Map is a simple interface to create parallelism of an identical function, e.g.,  a photo management application performs the same "compression" function on
	each of the images in a unzipped folder.  The map pattern invokes the same
function, once for each element, in a vector of outputs from a single source
function.
	\item \textit{\textbf{fan in:}} is a common pattern to join back multiple parallel functions, e.g., a video encoder that has encoded video chunks each in parallel might want to concatenate all the encoded
	chunks together. The fan-in pattern invokes a single ``sink''
	function with the outputs from a vector of functions (the fan-in branches).
	
	\item \textit{\textbf{branch:}} \shadiS{ is a new pattern supported by \name{} where any  transition pattern can also be conditional, i.e., only execute when the conditions are met. Branch is simple construct with powerful implications, e.g.,  \shadi{@David: example? something in line of "selecting the ML model to run on an image based on the size of the image"}. It also it allows building complex patterns such as ``while'' loops and recursion by using branch as the condition to end the while-loop or recursion The
	branch pattern selects a next function based on some boolean
	condition.
}

	\item \textit{\textbf{fold:}}  \shadiS{is an advanced pattern that is supported by few
	systems~\cite{azure-functions}. Fold sequentially applies the same function on the outputs of a vector of functions, while aggregating with the result of running the function outputs so far. An example of fold would be concatenating several video chunks in order: concatenating chunk 1 and 2, then concatenating chunk 3 to chunk 1+2, and so on.
	}
	
\end{itemize}

These four patterns  can express a rich variety of workflows
efficiently, including a superset of workflows expressible in AWS Step
Functions and all workflows we encountered. In the next section we describe how we execute these pattern with no orchestrator involved.


\begin{figure}[t!]
	\centering
	\scalebox{.9}{\includegraphics[width=\columnwidth]{figures/excamera.pdf}}
	\caption{ExCamera breakdown of workflow transition patterns. }
	\label{fig:excamera}
\end{figure}


%\name{} supports four general  control-flow/transition patterns: \shadi{do we want to use the name "transition" or "control-flow". Lets be consistent throughout.}
%\begin{itemize}
%	\item \textit{chain} ($f\rightarrow g$): where $g$ is invoked with the output of $f$. Chaining is a simple but most fundamental control-flow pattern, e.g.,  a function processing sensor data followed by another function adjusting an
%	actuator. 
%	\item \textit{map}($f[o_1, ..., o_n]\rightarrow g$): where $g$ is invoked on  \textit{multiple} outputs of $f$. Map is a simple interface to create parallelism but on an identical function, e.g.,  a photo management application performs the same "compression" function on
%	each of the images in a unzipped folder. 
%	\item \textit{fan out} ($f\rightarrow [g_1, ..., g_n]$): where and array of (identical or different) functions, $g_1$ to $g_n$ are invoked in parallel with the output of $f$. Fan-out is a common pattern to create parallel processing,  e.g.,  an social network application performing
%	several independent functions given a new user post, such as URL shortening
%	and resolving other users mentioned in the post. 
%	\item \textit{fan in} ($[f_1, ...f_n] \rightarrow g$): where the output of $n$ (identical or different) parallel functions are aggregated and then $g$ is invoked on the array of \textit{all} outputs. This is a common pattern to join back multiple parallel functions, e.g.,\shadi{??}. 
%\end{itemize}
%
%\name{} applications/workflows are built by using arbitrary composition of  these patterns, e.g., a serverless video encoder that divides a large
%video into chunks(\shadi{?}), encodes each chunk in parallel (map) and concatenates all back
%at the end (fan-in).
%These four patterns  can express a rich variety of workflows
%efficiently, including a superset of workflows expressible in AWS Step
%Functions and all workflows we encountered. In the next section we describe how we execute these pattern with no orchestrator involved.


%
%\subsection{Decentralized execution of transitions}
%\label{sec:transition-execution}
%
%A key challenge in decentralizing control-flow is determining where to store
%control-flow state and how to execute transitions, without a global view of the orchestrator. \name{} decentralized each pattern by logically partitioning the transition into a pair of nodes:  an \textit{egress node(s)} appended to the
%upstream function and a matching \textit{ingress node(s)} prepended to the immediate
%downstream function(s).  Every user function is invoked once with a
%single value by an ingress node and outputs its result to a single egress
%node.  Figure~\ref{fig:transition} demonstrates the ingress/egress nodes added for each pattern. Each ingress/egress node performs a specific logic on the incoming/outgoing data depending on the transition pattern,  as we will discuss further below.  We note that transition patterns only uses the basic serverless abstractions, universally supported by current platforms, and are platform agnostic.
%Specifically, they relies
%on an asynchronous invocation API for FaaS functions and a strongly consistent
%data store that supports conditional writes (for fan-in and exactly-once guarantees only). 
%
%%Logically, each pattern is designed as
%%a pair of nodes: an ingress node and an egress node. The egress node is appended to the
%%upstream function and the matching ingress node is prepended to the immediate
%%downstream function(s).
%%Every user function in
%%\name{} has exactly one input edge coming from an ingress node and one output
%%edge going to an egress node, i.e., the user function is invoked once with a
%%single value by an ingress node and outputs its result to a single egress
%%node. 
%
%%The \name{} backend compiler converts the IR representation of functions  to platform specific \name{} runtimes with pairs of ingress/egress nodes (Figure \shadi{??}). The runtime then, based on the transition patterns embeded in \name{} config file, executes the patterns accordingly in each egress/ingress point. In this section we describe what exact execution logic does each pattern map to at the ingress and egress points. We note that \name{} runtime only uses the basic serverless abstractions, universally supported by current platforms, and are platform agnostic.
%%Specifically, it relies
%%on an asynchronous invocation API for FaaS functions and a strongly consistent
%%data store that supports conditional writes (for fan-in and exactly-once guarantees only). 
%
%\begin{figure}[t!]
%	\centering
%	\scalebox{.7}{\includegraphics[width=\columnwidth]{figures/excamera.pdf}}
%	\caption{TODO}
%	\label{fig:excamera}
%\end{figure}
%
%\noindent\textbf{Chain:}
%The chain pattern invokes a single target function with the result of a source function.
%As shown in Figure~\ref{fig:transition}, chaining consists of one egress node 
%appended to the source function and one ingress is prepended to the target.
%At runtime, the egress node gets the output of the source user function
%and uses the platform's asynchronous function invocation API to call the
%target function directly from the source. The ingress on the target reads the
%data sent from source and passes it to the target user function. Depending on
%the platform's implementation of asynchronous invoke API, the ingress might
%read the input data from a data store or the received HTTP message.
%
%
%\noindent\textbf{Fan-out:}
%The fan-out processes the output of a function  in parallel, by 
%invoking vector of functions (branches) each with the output of the same
%source function. The fan-out pattern consists of one egress node and many ingress
%nodes (one per branch). Similar to chaining, the egress node runs with the
%source function and it uses the asynchronous invocation API to invoke each
%branch where each ingress node 
%reads the input data sent from the source and passes it to its user function.
%
%
%\noindent \textbf{Map:}
%The \texttt{map} gadget invokes the same
%function once for each element in a vector of outputs from the source
%function. The algorithm and placement of \texttt{map} ingress and egress nodes
%are identical to \texttt{fanOut}. \dhl{TODO: talk about the dynamic aspect of Map}
%\shadi{Map seems very empty. why do we have it separated? any details and challenges to add here? }
%
%
%
%
%\noindent\textbf{Fan-in:} The \texttt{fanIn} pattern takes the outputs from a vector of
%functions (the fan-in branches) and invokes a single ``sink'' function. As shown in Figure~\ref{fig:transition}, fan-in consists of one ingress prepended to the sink function and several egress nodes, each
%appended to a fan-in branch. Fan-in has an important requirement--the sink function should be called \textit{once} and only when \textit{all} fan-in branches have completed. At the same time, \name{} aims to avoid idle-billing and therefore extra billing (Section \shadi{??}). Thus, the fan-in pattern should be ``wait-free'' on both sides: a) the upstream functions should terminate after completion and not wait for each other, and b) the sink function should not  spin up ahead of time idly waiting for upstream functions to finish.
%
%% TBD
%%
%% The \texttt{fanIn} gadget gadget ensures that the control-flow transition is
%% \emph{wait-free} and that the sink function is invoked only once.
%% Specifically, its semantics is that the sink function is invoked only once
%% when all upstream functions in the vector have completed.
%
%% To realize the semantics, the \texttt{fanIn} gadget has to solve several
%% challenges. Access to branches data. wait-free. and synchronize.
%
%%The \texttt{fanIn} gadget ensures that the control-flow transition is
%%\emph{wait-free}. That is the sink function is invoked only when all upstream
%%functions in the vector have completed so that the sink function does have to
%%be spun up ahead of time and waste CPU cycles (and therefore extra billing)
%%idly waiting for upstream functions to finish. Moreover, the upstream
%%functions in \texttt{fanIn} simply terminates when done and do not wait for
%%each other either.
%
%To achieve this, \name{} uses an intermediary data store (as shown in Figure \shadi{??}) to log the output of upstream functions and signal the completion to the sink. In this design, each egress  node simply
%writes its output and terminates the function (avoiding idle-billing on the upstream functions), with the exception of the ``last'' egress node which first invokes the sink function. \name{} synchronizes among the upstream functions to identify the ``last'' function using an
%atomic read-after-write over a single object. Every upstream function \shadi{....? give a high-level idea. what is the process here, a few senteces in the data-store independent manner!}.  This ensures when the sink function is
%invoked, it is invoked only once.  Specific implementation depends
%on the data store as we discuss the details in \S\ref{sec:impl}.  Given that the data store is shared among all upstream functions, any egress node can see if another egress node has completed and any egress can invoke the sink function. 
%
%The ``last'' egress invokes the sink function with a vector of pointers
%to each upstream function's stored output. The pointers are the in same order
%as the vector of upstream function names.\shadi{does this point regarding the naming matter? if so, add some details. If not, maybe remove?} The ingress on the sink function
%dereferences each point by reading from the data store and passes a vector of
%output values to its user function.
%
%
%We have implemented \name{} with both Dynamo DB and S3. However, \name{} can be integrated with any data-store as long is it provides strong consistency with conditional writes. Strongly consistent is required to prevent the
%scenarios where all egress nodes have written outputs but none of them sees that all
%have completed, which will result in the sink function never being invoked. \shadi{wouldn't eventual be enough? the sink will eventually be invoked}
%\shadi{give insight why you need conditional writes. this comes out of the blue}. 
%\dhl{?TODO: Say more about synchronization? That it needs to be idempotent
%	because functions can crash and be retried.}
%
%
%%To achieve this, the \texttt{fanIn} egress always writes the output of its
%%user function to a data store. This serves two purposes: (1). it allows any of
%%the upstream functions to access the output of other upstream functions (2).
%%it signals the completion of a function. This way, each egress can simply
%%writes its output and terminate. 
%%
%%Other egress nodes can still access completed
%%egress' data after they terminate. Any one of the egress can invoke the sink
%%function. And any one of the egress can see if other egress has completed or
%%not. \dhl{definitely needs better phrasing but hopeful the point makes sense}
%
%
%
%
%
%
%\dhl{?TODO: Fan-in is a critical control-flow pattern to support and distinguishes \name{}
%	from ad-hoc trigger-based composition.  Do we want to constrast here? and what
%	should we say? Different from ad-hoc compositions: 1. use of data store 2.
%	Control flow logic not only on the caller but also on the callee 3. standard,
%	off-the-shelf primitives that's not application specific that developers build
%	from scratch.}
%
%\dhl{?TODO: Fan-in expresses data dependencies. More flexible/expressive than
%	Step Functions because SF only supports dependencies within a state. unum
%	fan-in can specify any function in the workflow.}
%
%
%
%
%
%
%
%
%
%
%
%
%
%\shadi{:::::::::left over comments. not sure if they need to be addressed?}
%
%
%
%%
%%\amit{TODO: I feel like there is a bunch of complexity, particularly with
%%	fan-in, do with assigning indexes to branches, etc, that is part of the
%%	platform-agnostic design of unum, so should be here. But I don't recall the
%%	specifics. Are there also similar things for the other gadgets?}
%%\dhl{I'm not sure what you mean by "similar things". But the branch indexes
%%	are assigned by the fan-\emph{out} node. The purpose is to give each node in
%%	the graph a unique name. I really don't think we should discuss the naming
%%	aspect in this section. I think this structure of writing the design works
%%	very well if we keep the gadgets to be general algorithms for control-flow
%%	patterns that are designed against an abstract serverless machine. We can talk
%%	about what we require from the serverless abstraction, but we shouldn't talk
%%	about the naming scheme. The gadget just cares that each function has a name,
%%	that you can identify them. It does not care how. Then in the IR section, we
%%	can talk about how the IR actually encodes the gadgets, and that's where we
%%	can explain that (1). we need each function to have a unique name for fan-in
%%	to work because we need to clearly identify which node's result to fan-in
%%	from, and we can have multiple instances of the same function in the graph
%%	because of fan-in and map. (2). our naming scheme is to assign an integer,
%%	starting with 0 and incrementing monotonically by 1, to each branch. And
%%	that's it. That's all the information we need at the IR level. And finally in
%%	the runtime section, we show the input payload which has a field for the
%%	branch index, and that's how we actually implement this piece of information.
%%	And we explain that the fan-out node adds this field to the payload when
%%	invoking each branch. It's like the storage layers where each layer adds a bit
%%	more information to enable specific additional functionalities.}
%
%
%
%
%% chain = invoke, fan-out = a bunch of invoke, one for each continuation;
%% Additionally, in the case of Map, one invoke for each element of the array
%% (output of the user function).  fan-in .... well.... let's see. The semantics
%% is: invoke the fan-in function once when all of its inputs are ready. In
%% practice, it is each branch synchronize over the data store to see if all
%% branches have completed. The last branch to complete calls invoke on the fan-in
%% function, and pass it pointers to all branches results/checkpoints in the data
%% store. The fan-in function first reads the branches' results, in order, via the
%% pointers, then pass them as input to the user code.
%
%\dhl{"strongly consistent data store with conditional writes". Is this going
%	to raise eyebrows when we later mention S3? Because technically, S3 does not
%	have a conditional write API. We implement it with its object versions
%	feature, which has to be turned on when creating the bucket. Maybe better to
%	call it something else.}
%
%% \dhl{I'm not sure it makes sense to treate fan-in specially on the directed
%	% graph level. In the previous version, the ingress gadget node on the fan-in
%	% doesn't really perform any \emph{control-flow logic}. It just reads the input
%	% data. And this is the same behavior across all gadgets. The ingress just read
%	% data, whether from a HTTP packet or from DynamoDB. You can pass a vector of
%	% pointers via a chain gadget and the ingress will do the same thing. But I
%	% guess more importantly, the ingress is simply and uniform enough that treating
%	% fan-in specially in our explanation doesn't add much value.}
%
%\dhl{From previous version: "At compile-time, \name{} injects these gadgets
%	into the nearest function and executes them in the \name{} runtime that wraps
%	the function. Thus there is no system overhead for running the gadgets---they
%	are, effectively, embedded in the functions themselves."The last sentence is
%	unclear to me. Running gadgets incur latency and additional Lambda billing.}
%
%%%%%%%%%%%%%%%%%%%
\subsection{\name{} Runtime}\label{sec:runtime}

% \begin{figure}[t!]
% 	\centering
% 	\scalebox{1}{\includegraphics[width=\columnwidth]{figures/deorc-patterns.pdf}}
% 	\caption{\deorc{} implements transitions in a decentralized manner where
% the egress on the source function(s) and ingress on the target function(s)
% work together to execute the orchestration. \shadi{does this figure add value given you have Fig 1. you can probably combine the patterns there ans name them there and reuse that figure.}}
% 	\label{fig:transition}
% \end{figure}

\begin{figure}[]
    \begin{minted}[
    frame=single,
    fontsize=\scriptsize
  ]{json}
{
    "Data": {
        "Source": "http | dynamodb | s3 | ...",
        "Value": "<object> | [<pointers>]"
    },
    "Session": "uuid",
    "Fan-out": {
        "Index": "int",
        "Size": "int",
        "OuterLoop": {
            "Index": "int",
            "Size": "int"
        }
    }
}
    \end{minted}
    \caption{\name{} runtime input payload schema}
    \label{fig:input-format}
\end{figure}


In this section, we describe how \name{} efficiently implements orchestration
and the six transition patterns in a decentralized manner. \name{} adds a
runtime library to each  user function to perform the transitions. The \name{}
runtime consists of an ingress and an egress component. During execution, the
egress on the source function(s) and ingress on the target function(s) work
together to execute the transition. The egress interprets its \name{}
configuration to learn what transition it needs to perform. The ingress reads
and formats input data sent by egress so that
\name{} orchestration logic can happen transparently without any change to the
user code.

A key requirement of \name{}'s design is to only use the basic serverless
abstraction without relying on any specialized APIs. Therefore, we build the
runtime such that it only depends on two serverless components that are
universally supported by all platforms: i.
asynchronous invocation of FaaS functions (e.g., as available in AWS Lambda, Azure Functions,
Google Cloud Funtions, Openwhisk) and ii. a strongly consistent data store
that supports conditional write operations (e.g., DynamoDB, Cosmos DB).

We describe how the \name{} runtime uses the two basic abstraction to support
each workflow pattern below:

\noindent\textbf{chain} 
The chain pattern involves one egress on source function and one ingress on
the target function. The source egress simply invokes the target function with
the source function's user code's output. \name{} uses asynchronous invocation
to avoid waiting and idle-billing.

When the target function is invoked, the input is first received by the
\name{} ingress. The ingress uses read the user function's input data, either
directly from the input payload or from a data store (e.g., if the
asynchronous invocation is via a storage trigger). Finally, ingress calls its
user function and passes it the input.

\noindent\textbf{fan-out}
The fan-out pattern involves one egress on the source function and many
ingresses on the target functions.
Similar to chaining, the egress asynchronously invokes each target and the
ingresses on targets read the input data sent from the source and passes it to
its user function.

\noindent\textbf{map}
map is a simple variation of fan-out where the egress treats its user code
output as an iterable. For each element in the iterable, the source egress
invokes the same target function to launch a unique runtime instances and pass
it the element.

\noindent\textbf{branch}
Branch is similar to fan-out except that each branch has a condition attached
and only executes when the condition evaluates to true. The egress first
evaluates the boolean condition before invoke the target function in that
branch. \shadi{is branch similar to fan-out or chain?}

\textbf{fan-in}
The fan-in pattern involves one ingress node and many egress nodes.
It is the main complexity of decentralized orchestration as we want to ensure
that the transition is \emph{wait-free} to avoid idle-billing. In particular,
we want to invoke the sink function only when all upstream functions have
completed so that the sink function does not have to be spun up ahead of time and
wait for upstream functions to finish. Moreover, the upstream functions should
simply terminate when done instead of waiting for each other.

To achieve this, \name{} relies on a strongly consistent data store with
conditional writes. \name{} egress always writes the output of the fan-in
source functions to an intermediary data store. The write not only signals the
completion of a function, but also allows any of the source functions in a
fan-in to access the output of each other. This way, each egress can simply
write its output and terminate, and the last-to-finish source function will
invoke the sink function.
\name{} requires this data store to be ``strongly consistent''. This is needed
to prevent the scenarios where all egress have written outputs but none of
them sees that all have completed, which results in the sink function never
invoked.

Additionally, fan-in makes sure that the sink function is only invoked once.
\name{} achieves this by having the egress nodes synchronize with each other
via intermediary data store such that only the last-to-finish egress invokes
the sink function. Synchronization is done with atomic read-after-write over a
single object, which requires the conditional write ability from the data
store. Specific implementation depends on the data store and we discuss the
details in \S\ref{sec:impl}.

The last-to-finish egress invokes the sink function with a vector of pointers
to each upstream function's stored output. The pointers are the in same order
as the vector of upstream function names. The ingress on the sink function
dereferences each point by reading from the data store and passes a vector of
output values to its user function.


\subsubsection{Runtime Metadata}

To support a rich variety of orchestration patterns, \name{} requires a
specific input payload schema in JSON (Figure~\ref{fig:input-format}) that
contains \name{} runtime metadata. In particular, \name{} uses the
\texttt{Fan-out} field to store branch indexes. The \texttt{Fan-out} field
contains a recursive \texttt{OuterLoop} field that supports nested fan-outs.

The runtime additionally uses a \texttt{Session} field to support concurrent
invocations of the same workflow. The \texttt{Session} field is a UUID string
that is unique to a workflow invocation and shared by all constituent function
instances in the invocation. Function checkpoint names are prefixed by the \texttt{Session} string so that
concurrently invocations do not overwrite each other's data. We discuss the details on
\name{} checkpoints and execution guarantees in the Section~\ref{sec:exec-gntee}.


\subsection{\name{} Intermediate Representation}
\label{sec:ir}

The \name{} compiler converts high-level workflow definitions into an intermediate representation (IR). The IR couples each user function with a platform agnostic \name{} config file containing the details on how to transition from that function to the next function(s). Specifically, the \name{} config includes: i. \textit{which} function(s) to call next, ii. \textit{when} to call the next function(s), iii. \textit{what} to send to the next function(s). 



In practice, the \name{} config of a particular transition pattern is  is a set of static configuration
files written in JSON. The compiler drives these files, but developers can also directly write these files. 
There should be one configuration file for each transition and it
is placed in the source function. During
execution, the \name{} runtime reads the file and performs control-flow
transitions to the next function(s) based on the configuration.


Figure~\ref{fig:arch2} shows examples for each transition type encoded
by the \name{}~IR. The \texttt{Name} field identifies the source function, which is also the function where the
configuration file is placed. The \texttt{Next} field identifies the function
to which the destination function(s) and the
\texttt{InputType} field helps define the transition type. The value
\texttt{Scalar} instructs the runtime to treat user function's output as a
single entity, and when the \texttt{Next} field contains only one object
(i.e., not an array), it represents a \texttt{chain} pattern; whereas when the
\texttt{Next} field is an array, it represents a \texttt{fanOut} pattern.
\texttt{map} pattern have a special \texttt{Map} value for the
\texttt{InputType} field as it instructs the runtime to expect and treat the
output of the user function as an iterable. \shadi{add sentence on fan-in}.






\begin{figure*}[t!]
	\centering
	\begin{subfigure}[t]{\columnwidth}
		\centering
		\begin{minted}[
			frame=single,
			fontsize=\scriptsize
			]{json}
			{
				"Name": "F",
				"Next": {
					"Name": "G",
					"InputType": "Scalar"
				}
			}
		\end{minted}
		\caption{\texttt{chain} pattern that invokes function \texttt{G} with
			\texttt{F}'s result}
		\label{fig:gadget-examples-chain}
	\end{subfigure}
	\begin{subfigure}[t]{\columnwidth}
		\centering
		\begin{minted}[
			frame=single,
			fontsize=\scriptsize
			]{json}
			{
				"Name": "F",
				"Next": {
					"Name": "G",
					"InputType": "Map"
				}
			}
		\end{minted}
		\caption{\texttt{map} pattern that invokes a parallel instance of
			\texttt{G} for each element of the vector output of \texttt{F}}
		\label{fig:gadget-examples-map}
	\end{subfigure}
	\hfill
	\begin{subfigure}[t]{\columnwidth}
		\centering
		\begin{minted}[
			frame=single,
			fontsize=\scriptsize
			]{json}
			{
				"Name": "F",
				"Next": [
				{
					"Name": "G",
					"InputType": "Scalar"
				},
				{
					"Name": "H",
					"InputType": "Scalar"
				}
				]
			}
		\end{minted}
		\caption{\texttt{fanOut} pattern that invokes function \texttt{G} and
			\texttt{H} with the result of \texttt{F}}
		\label{fig:gadget-examples-fanout}
	\end{subfigure}
	\begin{subfigure}[t]{\columnwidth}
		\centering
		\begin{minted}[
			frame=single,
			fontsize=\scriptsize
			]{json}
			{
				"Name": "F",
				"Next": {
					"Name": "G",
					"InputType": {
						"Fan-in": {
							"Values": [
							"F-unumIndex-*"
							]
						}
					}
				}
			}
		\end{minted}
		\caption{\texttt{fanIn} pattern that invokes function \texttt{G} with
			the result of all \texttt{F} instances of a \texttt{map}.}
		\label{fig:gadget-examples-fanin}
	\end{subfigure}
	\caption{The IR representation of the various transition patterns of \name{} . \shadi{can we shrink this figure? maybe reduce font size and put all in one row (instead of two)}}
%		 System Overview. Serverless computations form a directed
%		graph that encode sequential and data dependencies between functions. Workflow
%		orchestrators drive these graphs by centralizing control flow logic and
%		interposing on all communication between functions. \name{},
%		instead, decentralizes control flow logic among the functions with
%		no need for a separate orchestration system.}
	\label{fig:arch2}
\end{figure*}

%%%%%%%%%%%%%%%%%%%%%%%


\subsubsection{Dynamic runtime behavior}\label{sec:ir:naming}


A key design challenge of the \name{} IR is to support dynamic runtime
behavior with statically generated configurations.
% For instance, the number of
%parallel branches in a \texttt{map} depends on the output of the egress node
%and cannot be known at compile-time. 
For instance, a workflow that consists of a \texttt{map}  followed by a
\texttt{fanIn} the number of parallel branches depends on the output of
the \texttt{map} egress node and cannot be known at compile-time. Furthermore,
\texttt{map} creates multiple instances of the same function. \name{} needs to
uniquely identify each \emph{runtime} instance so that \texttt{fanIn} can
execute correctly (e.g., not miss an instance or counting the same instance
twice).

To solve this problem, \name{} defines a naming scheme for runtime instances
in the control-flow graph and exposes a set of programming constructs. First,
\name{} requires each user function to have a user-defined name. This is also
a common requirement for existing serverless systems when developers deploy
their functions. Next, each branch in a \texttt{map} and \texttt{fanOut}
gadget is assigned an integer index, starting from zero, and the $i^th$ branch
is named \texttt{<FunctionName>-unumIndex-\emph{i}}. For nested fan-outs, the
indexes are delimited with periods. For example,
\texttt{<FunctionName>-unumIndex-\emph{i.j}} identifies the $i^{th}$ branch in
the outer loop and $j^{th}$ in the inner loop.

To identify all branches, the \name{}~IR supports glob patterns, such as
\texttt{*}, when specifying runtime instances' names.
Figure~\ref{fig:gadget-examples-fanin} shows a \texttt{fanIn} example that
invokes \texttt{G} with the outputs from all instances of \texttt{F}.

\subsubsection{Programmable Constructs \shadi{better name?}}

\begin{figure}[]
	\begin{minted}[
		frame=single,
		fontsize=\scriptsize
		]{json}
		{
			"Name": "F",
			"Next": [
			{
				"Name": "G",
				"InputType": "Scalar",
				"Conditional": "$ret < 0"
			},
			{
				"Name": "H",
				"InputType": "Scalar",
				"Conditional": "$ret >= 0"
			}
			]
		}
	\end{minted}
	\caption{\texttt{F} branches on the user function's result by
		combining \texttt{fanOut} gadget with \texttt{Conditional}}
	\label{fig:gadget-examples-branch}
\end{figure}

In addition to the naming scheme, the IR also provides a set of constructs
that directly controls dynamic behavior. The complete set of programmable
constructs are beyond the scope of this paper, but we highlight one
API---\texttt{Conditional}--- that enables
branching. Figure~\ref{fig:gadget-examples-branch} shows an example of using
the \texttt{Conditional} field to control whether to run a pattern based on the
user function's output (\texttt{\$ret}). Combining \texttt{Conditional} with
\texttt{fanOut}, we can express branching logic in the control-flow. We can
	also express \texttt{for} loops and \texttt{fold} patterns (\shadi{i.e., ?}) using recursive function calls and conditionals as the termination
	condition. \shadi{can we have these in the example figure above?}

We highlight this example because it showcases the extensibility of \name{}'s
design: Given a small set of gadgets, \name{} can layer on top of them to
enable more dynamic control-flow logic.






\subsection{Execution Guarantees}\label{sec:exec-gntee}

An important characteristic of any workflow system is (a) how it deals with  a
transient failure in a constituent step, and (b) what guarantees it makes in
the presence of such faults.
 
Workflow systems typically persist progress to limit the scope of re-execution after faults
\cite{aws-step-functions, durable-functions, netherite, google-workflows, kappa}.
Likewise, \name{} checkpoints each function to storage.
 In particular, if a workflow experiences crashes mid-execution,
\name{} does not retry from the beginning but from the node of failure only.

In terms of progress and consistency, \name{} guarantees \textbf{exactly-once semantics}, 
meaning that the system records
exactly one result for each step of the workflow. FaaS engines already
support automatic retries for functions; this helps to ensure progress, as transient
faults do not compromise progress. However, we still need to
strengthen this at-least-once guarantee to an exactly-once guarantee, which 
is nontrivial because of the following subtleties:

\squishlist
	\item Function executions are not always deterministic, each re-execution
	may produce a different result.
	\item Some FaaS engines may detect failures incorrectly, thus multiple
	executions of a function can be in progress simultaneously, and may all run to completion.
\squishend
\vspace{1ex}

Fortunately, we found a way to handle these challenges by taking advantage of
conditional store operations supported by strongly consistent data stores.
Specifically, \name{} guarantees that even if there are multiple function
execution instances, concurrent or not, only one instance's result is taken as
the final result and propagates to the downstream ingress node(s). Other
instances simply discard their results and terminate.

\paragraph{Checkpoints and Synchronization.}%\name{} uses a similar checkpointing technique across all transition patterns. 
After user code completes, the \name{} egress immediately writes a checkpoint
file that contains the user code results to the intermediary data store. The
checkpoint is uniquely named with the instance's name (i.e., the name
according the
\name{}~IR's naming scheme (\S\ref{sec:ir:naming}), prefixed by the workflow
invocation's unique session ID) such that the existence of a checkpoint
implies the corresponding function has successfully completed its user
function. The create operation is a conditional write and only succeeds when
the file does not already exist. If there are concurrent duplicate instances,
only one of them will create the checkpoint. The others will receive an error
from the write operation and \name{} runtime will simply terminate the
instance. The instance that successfully creates a checkpoint will proceeds to
executing its egress node and propagate its result to downstream functions.

For nonconcurrent duplicates (e.g., retries), \name{} checks if a checkpoint
exists \emph{before} running its user code. If a checkpoint does not exist,
\name{} goes ahead and runs the user code. Otherwise, \name{} reads the data
from the checkpoint and use that as final result without running user code
again. Then \name{} will run the ingress node to invoke the downstream function.
This is necessary because the duplicate might be a retry whose prior execution
crashed after checkpointing but before running the downstream ingress node. \name{} can
tolerate running a ingress/egress node more than once because of the same protection
against duplicates.

\paragraph{External Side Effects.} Naturally, the
exactly-once guarantee does not automatically extend to functions with
external side effects, i.e. functions that directly call external services. In
such cases, retries can lead to unexpected results if the effects are not
idempotent. This issue is well known, and independent of the orchestrator
architecture (centralized vs. decentralized). Thus, we consider the question
of how to control such side effects to be orthogonal and beyond the scope of
this paper. For example, a store interposition libary like Beldi \cite{beldi}
can solve this problem.
