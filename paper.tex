\documentclass[letterpaper,twocolumn,10pt]{article}
\usepackage{usenix2019_v3}
\usepackage{dirtree}
\usepackage{times}
\usepackage{amsmath}
\usepackage{amsfonts}
\usepackage{pifont}
\newcommand{\cmark}{\ding{51}}%
\newcommand{\xmark}{\ding{55}}%
\usepackage{adjustbox}    % Auto-resize table content
%\usepackage{booktabs}
\usepackage[outputdir=latex.out]{minted}
\usepackage{minted}
\usepackage{graphicx}
\usepackage{listings}
\usepackage{subcaption}
\usepackage{framed}
\usepackage{soul}
\usepackage{textcomp}     % Provides \textmu for upright mu's
\usepackage{tikz}
\usetikzlibrary{positioning}
\usepackage{xspace}
\usepackage{tabularx}
\frenchspacing

% listing styles
\definecolor{codegreen}{rgb}{0,0.6,0}
\definecolor{codegray}{rgb}{0.5,0.5,0.5}
\definecolor{codepurple}{rgb}{0.58,0,0.82}
\definecolor{darkgreen}{rgb}{0.2,0.4,0}
% \definecolor{backcolour}{rgb}{0.95,0.95,0.92}

\usepackage{courier}

\lstdefinestyle{mystyle}{
    % backgroundcolor=\color{backcolour},   
    commentstyle=\color{codegreen},
    keywordstyle=\color{magenta},
    numberstyle=\tiny\color{codegray},
    stringstyle=\color{codepurple},
    basicstyle=\ttfamily\footnotesize,
    breakatwhitespace=false,         
    breaklines=true,                 
    captionpos=b,                    
    keepspaces=true,                 
    numbers=left,                    
    numbersep=5pt,                  
    showspaces=false,                
    showstringspaces=false,
    showtabs=false,                  
    tabsize=2
}
\lstset{style=mystyle}

% And, of course, cleveref must be loaded last-last (read: after hyperref)
\usepackage[capitalise,nameinlink,noabbrev]{cleveref}     % Do the right thing with fig/table references

% Break URLs properly (thanks to Alex Halderman)
\def\UrlBreaks{\do-\do\.\do\@\do\\\do\!\do\_\do\|\do\;\do\>\do\]\do\)\do\,\do\?\do\'\do+\do\=\do\#}
\def\UrlBigBreaks{\do\:\do\/}

%%
\def \gg {\texttt{gg}\xspace}

% comment first line for submission, second line for draft
% \newcommand{\note}[1]{#1}
\newcommand{\note}[1]{}

\newcommand{\shadi}[1]{ {\note{\bf \color{cyan} Shadi: #1}} }
\newcommand{\shadiS}[1]{ {\note{\bf \color{blue} #1}} }
\newcommand{\dhl}[1]{ {\note{\color{red} DHL: #1}} }
\newcommand{\amit}[1]{ {\note{\color{purple} Amit: #1}} }
\newcommand{\seb}[1]{ {\note{\color{darkgreen} Seb: #1}} }


\newcommand{\squishlist}
{
 \vspace*{-0.1cm}
 	\begin{list}{$\bullet$}
		{
			\setlength{\itemsep}{0pt}      \setlength{\parsep}{3pt}
			\setlength{\topsep}{3pt}       \setlength{\partopsep}{0pt}
			\setlength{\leftmargin}{1.5em} \setlength{\labelwidth}{1em}
			\setlength{\labelsep}{0.5em}
		}
	}
	
\newcommand{\squishend}
	{
	\end{list}  \vspace*{-0.2cm}
}


 \newcommand{\squishenum} %Indy
 {
	      \vspace*{-0.2cm}
	      \begin{enumerate}
		     {
			         \setlength{\itemsep}{0pt}
			         \setlength{\parsep}{3pt}
			         \setlength{\topsep}{3pt}
			         \setlength{\partopsep}{0pt}
			         \setlength{\leftmargin}{1.5em}
			         \setlength{\labelwidth}{1em}
			         \setlength{\labelsep}{0.5em}
			     }
		 }
	
	 \newcommand{\squishenumend}
	 {
		     \end{enumerate} \vspace*{-0.2cm}
	 }


\newcommand{\name}{\texttt{unum}}
\newcommand{\sfnc}{Step Functions}

%-------------------------------------------------------------------
\begin{document}
%-------------------------------------------------------------------

\date{}

% make title bold and 14 pt font (Latex default is non-bold, 16 pt)
% \title{\Large \bf unum: A Library Workflow System for Serverless Applications}
\title{\Large \bf unum: Purely Serverless Workflows with Distributed
Orchestration}


%for single author (just remove % characters)
\author{
% {\rm David H. Liu}\\
% Princeton University
% \and
% {\rm Amit Levy}\\
% Princeton University
% \and
% {\rm Shadi Noghabi}\\
% Microsoft Research
% \and
% {\rm Sebastian Burckhardt}\\
% Microsoft Research
% % copy the following lines to add more authors
% % \and
% % {\rm Name}\\
% %Name Institution
} % end author

\maketitle
\thispagestyle{empty}

\abstract
% Workflow systems have emerged to meet the demands of large serverless
% applications by providing high-level interfaces for workflow definition, rich
% semantics for function composition and strong guarantees for workflow
% execution.

Workflow systems have emerged to meet the demands of large-scale serverless
applications with higher-level programming interfaces, rich composition
semantics and strong execution guarantees. However, current designs rely on
supplemental hosted orchestrators to execute workflows and thus break out of
the serverless abstraction, compromising key serverless benefits such as fast
autoscaling, fine-grained billing and efficient resource multiplexing.

This paper presents \name{}, a new design for serverless workflow systems that
supports existing higher-level interfaces while executes purely as serverless
functions. \name{} uses a two-stage compiler that distributes orchestration
logic to constituent functions at compile-time and removes the need for
orchestrators during runtime. \name{} uses checkpoints with strongly
consistent data stores to prevent re-executions and provide strong execution
guarantees.

We present and evaluate an implementation of \name{} that can compile Step
Functions state machines and execute them purely as Lambda functions with
either S3 or DynamoDB as the data store. Our comparison with Step Functions
shows that applications scale up to 4.58x faster and costs more than one
order-of-magnitude less when running with \name{}.

%-------------------------------------------------------------------
\section{Introduction}

\begin{itemize}

  \item (Serverless computing is getting increasingly
  popular~\cite{datadog-state-of-serverless} and is regarded by many as the
  next stage of the cloud computing evolution~\cite{berkeley}.)

  \item \textbf{Serverless provides many benefits compared with traditional
  architecture: }

    \begin{enumerate}

      \item Fast autoscale of functions. Serverless systems can quickly start
      thousands of functions in a matter of seconds.

      \item Fine-grained billing. Users only pay for the resources used and
      can avoid paying for resources when idle.

      \item More flexible resource management that improves multiplexing and
      resource utilization. Serverless providers can reallocate resources
      immediately after computation completes and scale an application down to
      zero when idle.

      \item (Decoupled computation and storage can scale and bill
      independently)

    \end{enumerate}

  \item Recently, there have been much interests in building large serverless
  applications that consists of multiple functions with complex orchestration
  patterns such as branching on function outputs, aggregating results over
  many functions, and launch a function on the completion of one or multiple
  upstream functions~\cite{excamera, kappa, pywren, google-workflows,
  durable-functions, gg-atc}. While the original serverless offerings provide
  an API for building single functions and simple mechanisms for routing
  function results, these applications require a system that has a high-level
  interface to express workflows of serverless functions and a backend that
  can synchronize and aggregate states across functions.

  \item (Many serverless workflow systems have emerged to meet this need.)

  \item \textbf{Unfortunately, these solutions all break out of the serverless
  paradigm and turn back to the traditional SaaS/PaaS architecture.}

  Workflows on these systems are written with a set of \emph{specialized APIs}
  and executed by \emph{purpose-built, centralized hosted services}
  (often called controllers, executors, coordinator or
  orchestrators)~\cite{gg-atc, excamera, kappa, triggerflow, pywren,
  durable-functions, aws-step-functions, google-cloud-composer,
  google-workflows}.

  \item \textbf{Requiring a hosted service outside the serverless environment
  is not only expensive but also fails to fully leverage the benefits of
  serverless.}

    \begin{enumerate}
      \item \textbf{Costs of an additional hosted service.} Adding additional
      services to the serverless environment takes significant effort and
      resources in terms of both developement and maintenance. Long-running
      orchestrators requires reserved resources and a team of engineers
      on-call.

      \item \textbf{Cannot leverage the benefits of serverless.} Relying on a
      centralized hosted service to drive workflows compromises important
      benefits of using a serverless system.

        \begin{enumerate}

          \item Scalability of workflows also depends on the performance of
          the orchestrator service. A slow orchestrator can become the
          bottleneck and nullify the fast autoscale advantage of serverless.
          Service providers need to make sure the orchestrator's performance
          match that of the serverless system.

          \item During execution, needs to run one orchestrator instance
          \emph{per workflow invocation} and the instance needs to stay around
          until all functions in the workflow complete, including idle time
          while waiting for functions to return results. This leads to reduced
          resource utilization as the service provider has to reserve the
          orchestrator's resources for the entire duration of the workflow
          even when the orchestrator is not doing useful work.

          \item Service providers pass the costs to users. Users are not
          billed in fine granularity solely for the resources used but also
          pay for orchestrators idle time (double billing).

          For examples, Step Functions bills based the number of state
          transitions (approx. the number of function invocations). Google
          Cloud Composer bills on orchestrator runtime on 10-min increments.
          In comparison, Google Cloud Functions are billed in 100ms
          increments. Azure Durable Functions uses a preempt and restore
          strategy to alleviate waiting and double billing, but users are
          billed whenever an orchestrator is restored (essentially you're
          billed for the occurance of idleness instead of the duration of
          idleness).

        \end{enumerate}

    \end{enumerate}

  \item \textbf{In this paper, we demonstrate a new approach to serverless
  workflows that are built purely within serverless systems.} Our approach
  transforms high-level representations of complex serverless workflows (e.g.,
  Step Functions) into functions that are executable on existing serverless
  platforms. With a runtime library, each function runs a subset of the
  workflow such that the collection of all functions execute the entire
  workflow without relying on any additional services.

  \item \textbf{High-level description of \name{}}

    \begin{enumerate}

      \item We build a system called \name{} that is capable of running
      complex workflows entirely within a serverless environment.

      \item \name{} consists of a suite of frontend compilers, a
      continuation-based intermediary representation and a runtime library.

      \item The frontend compilers can transform applications written for
      existing workflow systems (e.g., AWS Step Functions) into the
      continuation-based
      \name{} IR.

      \item We show that with the IR, we can distribute workflow actions onto
      component functions in an application. And a runtime library packaged
      with each function can then perform the assigned workflow action.

      \item \name{} supports stateful actions, such as fan-out, fold, with a
      named data store shared among functions.  We show that it is possible to
      manage runtime states with any storage \emph{as long as function outputs
      are uniquely named and the storage supports querying items with names
      (e.g., databases, KV stores, object stores, etc.)}.
    \end{enumerate}

  \item \textbf{Benefits}
    \begin{enumerate}
      \item Removes the effort to build and maintain specialized hosted
      workflow services
      \item Not being part of the underlying system, a library is easier to
      change, adapt and be portable.
      \item \emph{Ability to leverage the advantages of serverless:}
        \begin{itemize}
          \item Cheaper -> a proxy for resource utilization
          \item (Not sure about this one) Orchestration is automatically
          parallelized. Can potentially demonstrate if \name{} consistently
          outperforms Step Functions on highly-parallel applications.
          \emph{This is not to claim that Step Functions are fundamentally
          unscalable, but that because we build on top of a highly-scalable
          system (Lambda), the application just scales with that one Lambda
          API, instead of having to rely on an additional system (which is
          Step Function in this case) for its scalability.}
        \end{itemize}
    \end{enumerate}

  \item \textbf{Contribution}
    \begin{enumerate}
      \item Show that it is possible to support workflows entirely with
      serverless. --> \emph{Strengthens the case for serverless in general and
      explore the boundary of serverless capability}.
      \item (Result-dependent) Overhead of \name{} library is actually low. So
      this is not terrible idea from a performance standpoint.
    \end{enumerate}


\end{itemize}


\section{Background and Objectives}\label{sec:bg}

% \begin{table*}[]
% \centering
% \resizebox{\textwidth}{!}{

% 	% \begin{tabular}{|l|l|l|l|l|l|l|l|}
% 	% \hline
% 	%  & \textbf{Step Functions} & \textbf{Durable Functions} & \textbf{Beldi} & \textbf{Boki} & \textbf{gg} & \textbf{ExCamera} & \textbf{unum} \\ \hline
% 	% \textit{High-level workflow interface}                        & Y & Y & N & N & Y & Y & Y \\ \hline
% 	% \textit{Do not need supplemental services/ purely serverless} & N & N & Y & N & N & N & Y \\ \hline
% 	% \textit{Strong execution guarantee}                           & Y & Y & Y & Y & N & N & Y \\ \hline
% 	% \end{tabular}

% 	\begin{tabular}{|l|l|l|l|l|l|l|l|l|l|}
% 	\hline
% 	 &
% 	  \textbf{Step Functions} &
% 	  \textbf{Durable Functions} &
% 	  \textbf{Beldi} &
% 	  \textbf{Boki} &
% 	  \textbf{Cloudburst} &
% 	  \textbf{Kappa} &
% 	  \textbf{gg} &
% 	  \textbf{ExCamera} &
% 	  \textbf{unum} \\ \hline
% 	\textit{High-level workflow interface}                        & Y & Y & N & N & Y & Y & Y & Y & Y \\ \hline
% 	\textit{Do not need supplemental services/ purely serverless} & N & N & Y & N & N & N & N & N & Y \\ \hline
% 	\textit{Strong execution guarantee}                           & Y & Y & Y & Y & N & Y & N & N & Y \\ \hline
% 	\end{tabular}
% }
% \caption{Comparison with existing work}
% \label{table:positioning}
% \end{table*}


% Large-scale applications composed of many functions~\cite{excamera, gg-atc,
% deathstar} introduce several new challenges to serverless computing. 

In this section, we describe why it is hard to compose large serverless
workflows, why a purely serverless architecture is interesting and what our
objective are.

\subsection{Evolution of Serverless Workflows}

\begin{figure}[t!]
    \centering
    \scalebox{.4}{\includegraphics[width=\columnwidth]{figures/ChainExample.pdf}}
    \caption{Chaining two functions with unstructure composition, driver
    functions and orchestrators. In the unstructured composition approach,
	\texttt{F} asynchronously invokes \texttt{G} with \texttt{F}'s result. In
	the driver functions approach, the driver function first synchronously
	invokes \texttt{F} and after \texttt{F} returns, synchronously invokes
	\texttt{G} with \texttt{F}'s result. Even though orchestrators'
	programming interface hides low-level APIs such as \texttt{invoke}, it is
	architecturally similar to driver functions where the orchestrator invokes
	\texttt{F}, waits for \texttt{F} to return and then invokes \texttt{G}
	with \texttt{F}'s result.}
    \label{fig:chain-example}
\end{figure}

\subsubsection{Unstructured composition and driver functions}
While bringing many economical and operational benefits, the original
serverless abstraction is designed around individual functions and only
provides low-level APIs that can be used to compose functions.

There are two primary approaches that early adopters use to build larger
applications. The first approach is sometimes called unstructure
composition~\cite{netherite} where functions trigger each other via
\emph{asynchronous} invocations. The second is often called driver
functions~\cite{beldi} where a single function invokes other functions
\emph{synchronously}. Figure~\ref{fig:chain-example} depicts examples of
chaining functions with the two approaches.

Both approaches have important drawbacks. Unstructure composition scatters
control flow across constituent functions. As the number of functions
increases, development can quickly get unwieldy. Moreover, it does not support
important composition patterns such as fan-in, where we want to invoke a
function to aggregate the results of multiple upstream functions only after
all of them have completed.

Compared with unstructured composition, driver functions concentrate control
flow in a single function and supports aggregation. However, users pay for the
time when driver functions idly wait for callees to return. This causes
\emph{double billing}~\cite{double-billing}. Also, serverless platforms
commonly impose timeouts. A driver function instance has to wait until all
functions in the application are complete, which risks timeouts and limits
application scale.

Moreover, both approaches suffer from weak execution guarantees of the
underlying serverless system. Functions can crash mid-execution due to runtime
or hardware faults which may leads to automatic retries~\cite{}. Even in the
absence of faults, most serverless platforms only ensure at-least-once
execution~\cite{} so a single invocation can trigger multiple, potentially
concurrent, instances.

\subsubsection{Workflow orchestrators}

Recently, workflow orchestrators have emerged to tackle the challenges of
large serverless applications~\cite{excamera, gg-atc, aws-step-functions,
google-cloud-composer, google-workflows, durable-functions}. Orchestrators are
designed as separate hosted services that execute workflow definitions, invoke
constituent functions and manage workflow states. They offer (1). higher-level
programming interfaces that directly express function interactions and hide
low-level APIs, (2). a rich set of composition primitives, including
branching, chaining, fan-out and fan-in, (3). exactly-once semantics for
workflow execution, and (4). long or no runtime limits.

Unfortunately, there are several important drawbacks of the orchestrator
design that compromise key benefits of the serverless abstraction: (1).
End-to-end performance now also depends on the orchestrator service that is
separate from the FaaS engine. A slow orchestrator can become a bottleneck and
nullify the fast autoscale advantage of serverless (2). Architecturally,
orchestrators are similar to driver functions. Service provider cannot
immediately reclaim idle resources from orchestrators which limits
multiplexing and reduces utilization. (3). As a result of (2), service
providers pass the costs of the lost efficiency to users by employing
separately-designed pricing schemes that are coarse-grained and not
pay-for-what-you-use, essentially ``double-charging'' for idle orchestrator
resources that they cannot immediately reclaim. (4). Building and maintaining
a hosted service often requires a dedicated engineering team which is
expensive.

\subsection{Research question and objectives}

In this paper, we examine whether we can have the best of all three worlds. Specifically, we want a system that meets the following objectives:

\paragraph{Higher-level programming interface}

\paragraph{Supports all common interactions}

\paragraph{Exactly-once semantics}

\paragraph{Purely serverless} should not rely on hosted services outside the
serverless abstraction.

\paragraph{Avoids idle waiting} 





% Can we build a workflow system that allows dev to express workflows
% with higher-level interface, supports rich workflow patterns including
% aggregating over multiple functions, while executes purely within the
% serverless abstraction without external services and avoids double billing
% for idle cycles.

% Objective: would such a system be as fast as specially built orchestrators?



%------------------------

% While bringing many economical and operational benefits, the original
% serverless abstraction focuses on building and executing individual functions.
% Support for constructing larger applications that are composed of multiple
% functions is minimum. A key missing feature is an interface to program
% function interactions.

% As a result, early pioneers resort to manually invoking functions and managing
% shared application-wide states~\cite{hello-retail}. For instance, to chain two
% functions, say \texttt{F} and \texttt{G}, \texttt{F} needs to explicitly
% invoke \texttt{G} with \texttt{F}'s result in the source code. Invocations are
% done via asynchronous HTTP requests or storage triggers to avoid waiting. For
% example, Python functions can call Lambda's \texttt{invoke} API via HTTP using
% the AWS \texttt{boto3} sdk. Alternative, developers can configure an S3 bucket
% to generate events on certain storage operations (e.g., writes) that triggers
% \texttt{G}~\cite{aws-s3-event-announce}. Then build \texttt{F} such that it
% writes \texttt{G}'s input data to the pre-configured S3 bucket.

% This approach is often referred to as \emph{unstructured
% composition}~\cite{netherite} because the control flow is scattered across
% functions. Each constituent function contains only a part of the workflow
% logic in its source code.

% There are two main drawbacks of unstructured composition. First, developing
% and maintaining can quickly become unwieldy as the number of functions
% increases. Second, because the invoke API only passes the caller's data, it is
% difficult to support an important composition pattern--fan-in, where we want
% to invoke a function to aggregate the results of multiple upstream functions
% only after all of them have completed.



% Group driver functions and workflow systesm into orchestrators.

% Serverless workflow systems have emerged to solve these
% challenges~\cite{aws-step-functions, google-cloud-composer, google-workflows,
% durable-functions}.




% No higher-level interface to express function interactions. To chain two
% functions, say \texttt{F} and \texttt{G}, \texttt{F} needs to explicitly invoke
% \texttt{G} in its code (for example, Python Lambda functions can invoke other
% functions via Lambda's \texttt{invoke} API with AWS \texttt{boto3} sdk).

% Application states shared across functions have to be manually managed.



% Can we build a system that executes workflows written in with a higher-level
% interface but without a centralized orchestrator service that the platform has
% to host? 

%------------------------

% In the case of orchestrators/coordinators, functions call back to the
% orchestrator when complete and the orchestrator decides what to do next: which
% downstream function to invoke:

% mu: "mu uses a long-lived coordinator that commands and controls a fleet of
% workers."

% gg: "The main entry-point for executing a thunk is the coordinator program....
% Upon start, this program [the coordinator] materializes the target thunk's
% dependency graph... Then, the thunks that are ready to execute are passed to
% execution engines. ... When the execution of a thunk is done, the progra      m [the
% coordinator] updates the graph by replacing the references to the just-forced
% thunk ... The thunks that become ready to execute are placed on a queue and
% passed to the execution engines when their capacity permits."

% Step Functions: TBA


% unum's different: orchestration logic is distributed to each function. When a
% function complete, instead of calling back to a long-lived coordinator, the
% unum runtime on the Lambda decides what to do next. This design eliminates the
% need for a separate long-running service that executes workflows. Saves a
% communication trip back to the orchestrator.



% ----

% ExCamera or mu:

% all workers in mu use the same generic Lambda function that is capable of
% executing the work of any thread in the computation. (change the programming
% interface: developers not writing individual functions anymore. :"To design a
% computation, a user specifies each worker’s sequence of RPC re- quests and
% responses in the form of a finite-state machine (FSM), which the coordinator
% executes.")

% mu uses a long-lived coordinator that commands and controls a fleet of
% workers. The coordinator steps workers through their tasks by issuing RPC
% requests and processing responses. As examples, the coordinator can instruct
% the worker to re- trieve from or upload to AWS S3; establish connections to
% other workers via a rendezvous server; send data to workers over such
% connections; or run an executable.

% Do the output of one worker go through the coordinator to be sent to and
% consumed by another worker? No, there's a separate rendezvous server for
% worker communication. Like the coordinator, the rendezvous server is long
% lived. mu’s rendezvous is a simple relay server that stores messages from
% workers and forwards them to their destination.


% gg's Restrictions on user code: "gg thunks are designed to be deterministic."
% "it is not allowed to use the network or access unlisted objects or files." So
% functions are basically pure so that gg doesn't need to reason about
% exactly-once semantics.


% "In a long chain
% of rebasing, later threads spend much of their time waiting
% on predecessors (Figure 4). A more-sophisticated launch-
% ing strategy could save money, without compromising
% completion time, by delaying launching these threads."

% [This is another point about many of these frameworks, (I think cloudburst has
% this problem as well) that they need to launch custom-built containers *ahead
% of time*, not on-demand, not event-driven. Double billing becomes an issue.
% Utilization becomes an issue.]

% --- 
% Beldi and Boki.

% --- 
% Cloudburst

% ---

% DHL's Design notes:


% Motifs:

% gadgets are a set of general algorithms that implement control-flow patterns
% in a decentralized manner and can execute on constituent functions alongside
% user code.

% gadgets are platform-independent control-flow patterns that can be implemented
% efficiently in a decentralized manner.

% Control-flow patterns.

% IR describes control-flow "logic"



% %%%%% Notes


% Graphs:

% A graph that constrasts application \emph{execution} with \name{} and
% centralized orchestrator. It's ok for the application to be artificial but it
% should sufficient complex with chain, fan-out and fan-in. The orchestrator
% figure should show the orchestrator centralizing control flow and driving the
% execution. The \name{} graph should show functions with gadgets and zoom in,
% similar to Figure 2(c), to show the runtime wrapper and interactions with the
% data store. The centralized orhcestrator should have the same color with the
% gadgets. Different from the current (a), ingress should show gadgets too.

% A graph that show the compilation stage. It's important to show what we gain
% at compile time: assigning gadgets to nodes in the directed graph.

% Architecture%%%%%%%%%%%%%%%

% no new component. Instead, 

% a set of general algorithms, called ``gadgets'', that  implements control-flow
% patterns in a decentralized way using basic serverless APIs, 

% Similar to existing systems, \name{} takes the user write workflows in the
% form of a directed graph. Instead of an centralized orchestrator executing the
% graph, \name{} ... Figure~\ref{fig:arch:centralized} and \ref{fig:arch} constrasts.



% Computations are a direct graph.


% Describe the abstract machine (the serverless abstraction with async invoke +
% strongly consistent data store with conditional updates) and here in this
% section.

% "strongly consistent data store with conditional updates". Is this going to
% raise eyebrows when we mention S3? Because technically, S3 does not have a
% conditional update API. We implement it with its object versions feature,
% which has to be turned on when creating the bucket.



% Gadgets%%%%%%%%%%%%%%%%%

% Gadgets are the general algorithms that implements control-flow patterns in a
% decentralized way against a basic serverless abstraction that is universially
% supported by all serverless providers.

% A set of general algorithms that implements control-flow patterns in a
% decentralized way using basic serverless APIs.


% Different from ad-hoc compositions: 1. use of data store 2.
% Control flow logic not only on the caller but also on the callee 3. standard,
% off-the-shelf primitives that's not application specific that developers build
% from scratch.


% Then use the gadget section to describe the algorithm for each pattern.

% State that "\name{} provides four gadgets that map can express a rich variety
% of workflows efficiently, including a superset of workflows expressible in
% Step Functions". Keep the example applications for each pattern that's
% currently in the text to show the usefulness and necessity for each pattern.

% supports all commonly found patterns. In SF, and in DAGs.

% unum IR%%%%%%%%%%%%%

% intermediate representation language that encodes control-flow 

% The IR language primary encodes gadgets but also provides other programmable
% constructs that enables dynamic behavior at runtime. Have a table of
% programmable constructs? (Conditional, \$ret, \$0, \$size)

% In pratice the IR is a set of JSON file, one for each edge in the control flow
% graph transition from a function to the next in the control-flow.

% Each function node has a unique name.

% Scalar, Map, Parallel are pretty straightforward because it dosen't have data
% dependency on other functions in the workflow.

% Fan-in expresses data dependencies. More flexible than Step Functions because
% SF only supports dependencies within a state. unum fan-in can specify any
% function in the workflow.

% Programmable constructs: 

% Conditional: branch, termination condition for a loop, foward condition for a
% selective set of parallel branches



% Frontend compiler%%%%%%%%%%%%%%%%

% how to assign gadgets

% Runtime%%%%%%%%%%%%

% XXX: Transparently wraps around user code and interposes on entry and exit -> keep user code unchanged



% Show the runtime input payload structure


% Execution guarantees%%%%%%%%%%%%%%%%%%

% XXX: All unum runtime code needs to be idempotent

\section{System Architecture}\label{sec:architecture}

\begin{figure*}[t]
    \centering
    \begin{subfigure}[t]{0.8\textwidth}
        \includegraphics[width=\columnwidth]{figures/unum-arch-compile-time.pdf}
        \caption{Stateful serverless computations form a directed graph. Nodes
                are user defined FaaS functions and workflow ``gadgets'' that dictate the
                communication pattern between user functions.}
        \label{fig:arch:unum-compile-time}
        % unum injects gadgets to functions at compile time. But more
        % specifically, unum injects an encoding of gadgets at compile time.
        % The encoding expresses control-flow transitions just like what the
        % high-level workflow definition.
    \end{subfigure}
    \begin{subfigure}[b]{\columnwidth}
    \centering
        \includegraphics[width=0.8\columnwidth]{figures/unum-arch-centralized.pdf}
        \caption{A typical stateful serverless system drives workflow logic
                 using a centralized controller that manages the computation's state-machine.}
        \label{fig:arch:centralized}
    \end{subfigure}
    \hfill
    \begin{subfigure}[b]{\columnwidth}
    \centering
        \includegraphics[width=0.5\columnwidth]{figures/unum-arch-runtime.pdf}
        \caption{\name{} decentralizes controller logic by running local state
                 transitions alongside Faas functions inside the \name{} runtime
                library. There is no centralized controller, instead \name{} relies on existing
                serverless datastores such as DynamoDB or Cosmos DB.}
        \label{fig:arch:unum-runtime}
    \end{subfigure}
    \caption{\name{} System Overview. Serverless computations form a directed
            graph that encode sequential and data dependencies between functions. Workflow
            orchestrators drive these graphs by centralizing control flow logic and
            interposing on all communication between functions. \name{},
            instead, decentralizes control flow logic among the functions with
            no need for a separate orchestration system.}
    \label{fig:arch}
\end{figure*}

\dhl{TODO: Explain the purpose and use of the data store early on in the
design portion of the paper. Probably also a good idea to motivate the use of
data store in the background section.}


A key design feature of \name{} is to not introduce new components to the
basic serverless infrastructure. Instead, \name{} invents a set of general
algorithms, called ``gadgets'', that implement control-flow patterns in a
decentralized manner and can execute on constituent functions alongside user
code. Additionally, \name{} encompasses an intermediate representation
language that encodes gadgets and describes control-flow logic, a frontend
compiler that transforms higher-level workflow representations written by
developers to the intermediate representation, and a runtime library that
implements the gadgets with platform-specific APIs.

Similar to existing systems~\cite{aws-step-functions, google-workflows,
google-cloud-composer, gg-atc}, \name{} users define workflows using
high-level description languages, such as AWS Step Functions, that express the
control-flow in the form of directed graphs where nodes are serverless
functions and edges are control-flow transitions between functions. A
transition from function \texttt{F} to \texttt{G} involves (1). invoking an
instance of \texttt{G} and (2). passing \texttt{F}'s result to the \texttt{G}
instance.
\dhl{Describing the directed graph abstraction helps clarify and set the
mental model. Should also explain from the directed graph perspective in the
background:workflow orchestrators section so that this is not the first time
we say this. State machines (i.e., Step Functions) and DAGs (i.e., Google
Workflows, Apache Airflow, Dask) all fit into direct graphs.}

But different from existing systems where a centralized orchestrator executes
control-flow graphs by invoking every function and receiving their results
(Figure~\ref{fig:arch:centralized}), \name{} transforms this high-level
description to \name{}'s intermediate representation (IR) and embeds portions
of the control-flow logic to appropriate functions, \emph{at compile time}
(Figure~\ref{fig:arch:unum-compile-time}). Control-flow in the IR is encoded
as ``gadgets''---platform-independent control-flow patterns that can be
implemented efficiently in a decentralized manner. Finally, developers combine
their functions with a platform-specific \name{} runtime which implements
gadgets using platform-specific APIs and datastores.

During execution, the \name{} runtime performs control-flow transitions by
running its assigned gadget (Figure~\ref{fig:arch:unum-runtime}). The runtime
transparently interposes on user code entry and exit. On entry, the runtime
reads data sent by the caller and passes it to user code. On exit, the runtime
gets user code result and invokes the next function with it. Additionally, the
runtime implements a checkpointing mechanism that ensures exactly-once
semantics.

\dhl{say more about exactly-once semantics here?}


\amit{TODO: gadgets are the secret sauce, in particular noticing that fan-in needs to
be split between the source and target functions \emph{and} that source
functions must coordinate.}
\amit{TODO: gadgets don't run where programmers express them. Example with
fan-in. This is kind of the ``magic'' of the front-end compiler.}
\dhl{I'm not sure the above 2 statements are true. Maybe because I didn't
understand your "move" gadgets mental model and it can work. Let's chat.}



\section{Gadgets}\label{sec:gadgets}

A key challenge in decentralizing control-flow is determining where to store
control-flow state and where to execute transitions. \name{} uses the notion
of gadgets---a set of general algorithms for control-flow patterns that can be
implemented efficiently in a decentralized manner. At compile-time,
\name{} derives the necessary gadgets from a workflow definition written in
high-level description language and attaches them to the approprite user
functions. The \name{} runtime that wraps the user function implements the
gadgets with platform-specific APIs and executes its assigned gadget during
execution.

Logically, gadgets are treated as a special kind of node in the control-flow
graph and placed alongside user function nodes. Each gadget is implemented as
a pair of nodes: an ingress node and an egress node. Every user function in
\name{} has exactly one input edge coming from an ingress node and one output
edge going to an egress node, i.e., the user function is invoked once with a
single value by an ingress node and outputs its result to a single egress
node. When placing a gadget in the graph, the egress node is appended to the
upstream function and the matching ingress node is prepended to the immediate
downstream node along an existing edge.

\name{} provides four gadgets that can express a rich variety of workflows
efficiently, including a superset of workflows expressible in AWS Step
Functions. Table~\ref{tab:gadgets} lists the gadgets available in \name{} and
we explain each in detail in this section.

\name{} gadgets are designed against a basic serverless abstraction that is
universally supported by current platforms. The algorithms themselves are
platform-agnostic and use only basic serverless APIs. Specifically, they rely
on an asynchronous invocation API for FaaS functions and a strongly consistent
data store that supports conditional writes.

\dhl{"strongly consistent data store with conditional writes". Is this going
to raise eyebrows when we later mention S3? Because technically, S3 does not
have a conditional write API. We implement it with its object versions
feature, which has to be turned on when creating the bucket. Maybe better to
call it something else.}

% \dhl{I'm not sure it makes sense to treate fan-in specially on the directed
% graph level. In the previous version, the ingress gadget node on the fan-in
% doesn't really perform any \emph{control-flow logic}. It just reads the input
% data. And this is the same behavior across all gadgets. The ingress just read
% data, whether from a HTTP packet or from DynamoDB. You can pass a vector of
% pointers via a chain gadget and the ingress will do the same thing. But I
% guess more importantly, the ingress is simply and uniform enough that treating
% fan-in specially in our explanation doesn't add much value.}

\dhl{From previous version: "At compile-time, \name{} injects these gadgets
into the nearest function and executes them in the \name{} runtime that wraps
the function. Thus there is no system overhead for running the gadgets---they
are, effectively, embedded in the functions themselves."The last sentence is
unclear to me. Running gadgets incur latency and additional Lambda billing.}



\begin{table}[t!]
  \centering
  \begin{tabular}{ |m{8em}| m{13em} | }
    \hline
      \texttt{chain(a,b) }& Invokes function \texttt{b} with the output of \texttt{a} \\
    \hline
      \texttt{fanOut(a, [b])} & Invokes each element of \texttt{[b]} with the output from \texttt{a} \\
    \hline
      \texttt{map(a, b)} & Invokes \texttt{b} once for each element in the vector output of \texttt{a} \\
    \hline
      \texttt{fanIn([a], b)} & Invokes \texttt{b} once with the vector of outputs from each of \texttt{[a]} \\
    \hline
\end{tabular}
  \caption{\name{} workflows express complex interactions using a small set of
  gadgets. \texttt{a} and \texttt{b} are names that identify specific function
  instances in the control flow graph.}
  \label{tab:gadgets}
\end{table}

\dhl{pseudocode for each gadget? or a simple schematics showing the pattern
graphically?}

\paragraph{Chain}
Chaining is a simple but common control-flow pattern. For example, an
application might include a function (the source) that processes input data
from a sensor followed by another function (the target) that adjusts an
actuator based on the processed input. The \texttt{chain} gadget connects two
user functions together by invoking the target with the result of the source.
It consists of one egress node and one ingress node. The egress node is
appended to the source function while the ingress is prepended to the target.

At runtime, the egress node runs on the same FaaS function as the source and
ingress as the target. The egress gets the output of the source user function
and uses the platform's asynchronous function invocation API to call the
target function directly from the source. The ingress on the target reads the
data sent from source and passes it to the target user function. Depending on
the platform's implementation of asynchronous invoke API, the ingress might
read the input data from a data store or the received HTTP message.


\paragraph{Fan-out}
Another common interaction processes the output of a function in different
ways in parallel. For example, an social network application might perform
several independent functions given a new user post, such as URL shortening
and resolving other users mentioned in the post. The \texttt{fanOut} gadget
invokes a vector of functions (branches) each with the output of the same
source function. The gadget consists of one egress node and many ingress
nodes. Similar to chaining, the egress node is appended to and runs with the
source function and it uses the asynchronous invocation API to invoke each
branch. But in this case, an ingress node is prepended to each branch and
reads the input data sent from the source and passes it to its user function.

\paragraph{Map}
An application may also perform the same function on multiple outputs of a
source function. For example, a photo management application might unpack an
archive of high-resolution images in one function and perform compression on
each of the resulting images. The \texttt{map} gadget invokes the same
function once for each element in a vector of outputs from the source
function. The algorithm and placement of \texttt{map} ingress and egress nodes
are identical to \texttt{fanOut}.

\amit{TODO: Are there special considerations for exactly-once
semantics? i.e. is checkpointing different than it is in chaining?}
\dhl{No. In fact checkpoint works independently from the control-flow
patterns. The algorithm for exactly-once semantics is the same across all
gadgets.

Now specifically for fan-out it works like this: the fan-out initiator node
will checkpoint right after user code returns which saves the data that is
about to be fanned out. After checkpointing completes, the fan-out initiator
node invokes the branches. If it crashes at any point during the series of
invocations, unum retries the fan-out initiator lambda. The unum runtime on
the lambda will see that a checkpoint with its name already exists, and
therefore skips running the user function. But it will not skip retrying the
fan-out, and it restarts the fan-out \emph{from the beginning}, which will
result in duplicate instances for some or even all of the branches. But that
is OK. Because the unum runtime protects against duplicates and still ensures
exactly-once semantics. If the duplicates are concurrent (i.e., the original
branch instances are still running), we protects that with a conditional write
when checkpointing the results so only one instance of the duplicates will
win. If the duplicates are nonconcurrent (i.e., the original branch instances
already completed), the duplicates will skip running the user functions
altogether.

The fan-out initiator will keep retrying until a full fan-out is performed to
make sure at-least-once invocation.

% (explaining this makes me appreciate the consistency papers even more, 'cuz
% this stuff is hard to explain.)

So the patterns do not change the algorithms with which we provide exactly
once. They work independently from each other. The only real gotcha we need to
be careful with in \emph{implementing} exactly-once is nonidempotent
operations, as pointed out by you. Specifically, the synchronizations across
branches have to be idempotent. The purpose is prevent pre-mature fan-in.

Given the complexity of the exactly-once algorithm and the fact that it works
independently from gadgets, I think it works better if we have separate
section just for the execution guarantee and do not explain the details in
this section.}

\amit{TODO: I feel like there is a bunch of complexity, particularly with
fan-in, do with assigning indexes to branches, etc, that is part of the
platform-agnostic design of unum, so should be here. But I don't recall the
specifics. Are there also similar things for the other gadgets?}
\dhl{I'm not sure what you mean by "similar things". But the branch indexes
are assigned by the fan-\emph{out} node. The purpose is to give each node in
the graph a unique name. I really don't think we should discuss the naming
aspect in this section. I think this structure of writing the design works
very well if we keep the gadgets to be general algorithms for control-flow
patterns that are designed against an abstract serverless machine. We can talk
about what we require from the serverless abstraction, but we shouldn't talk
about the naming scheme. The gadget just cares that each function has a name,
that you can identify them. It does not care how. Then in the IR section, we
can talk about how the IR actually encodes the gadgets, and that's where we
can explain that (1). we need each function to have a unique name for fan-in
to work because we need to clearly identify which node's result to fan-in
from, and we can have multiple instances of the same function in the graph
because of fan-in and map. (2). our naming scheme is to assign an integer,
starting with 0 and incrementing monotonically by 1, to each branch. And
that's it. That's all the information we need at the IR level. And finally in
the runtime section, we show the input payload which has a field for the
branch index, and that's how we actually implement this piece of information.
And we explain that the fan-out node adds this field to the payload when
invoking each branch. It's like the storage layers where each layer adds a bit
more information to enable specific additional functionalities.}

\paragraph{Fan-in}

After processing data with many parallel branches, applications commonly want
to aggregate results. For example, a video encoder might divide a large video
into chunks, encode each in parallel and then concatenant all the encoded
chunks together. The \texttt{fanIn} gadget takes the outputs from a vector of
functions (the fan-in branches) and invokes a single ``sink'' function. The
\texttt{fanIn} gadget consists of one ingress node and many egress nodes. Each
fan-in branch is appended an egress node and the sink function is prepended
the only ingress node.

% TBD
%
% The \texttt{fanIn} gadget gadget ensures that the control-flow transition is
% \emph{wait-free} and that the sink function is invoked only once.
% Specifically, its semantics is that the sink function is invoked only once
% when all upstream functions in the vector have completed.

% To realize the semantics, the \texttt{fanIn} gadget has to solve several
% challenges. Access to branches data. wait-free. and synchronize.

The \texttt{fanIn} gadget ensures that the control-flow transition is
\emph{wait-free}. That is the sink function is invoked only when all upstream
functions in the vector have completed so that the sink function does have to
be spun up ahead of time and waste CPU cycles (and therefore extra billing)
idly waiting for upstream functions to finish. Moreover, the upstream
functions in \texttt{fanIn} simply terminates when done and do not wait for
each other either.

To achieve this, the \texttt{fanIn} egress always writes the output of its
user function to a data store. This serves two purposes: (1). it allows any of
the upstream functions to access the output of other upstream functions (2).
it signals the completion of a function. This way, each egress can simply
writes its output and terminate. Other egress nodes can still access completed
egress' data after they terminate. Any one of the egress can invoke the sink
function. And any one of the egress can see if other egress has completed or
not. \dhl{definitely needs better phrasing but hopeful the point makes sense}

Strongly consistent data store is important here because it prevents the
scenarios where all egress have written outputs but none of them sees that all
have completed, which results in the sink function never invoked.

Additionally, \texttt{fanIn} makes sure that when the sink function is
invoked, it is invoked only once. \name{} achieves this by having the egress
nodes synchronize with each other via the same data store such that only the
last-to-finish egress invokes the sink function. Synchronization is done with
atomic read-after-write over a single object. Specific implementation depends
on the data store and we discuss the details in \S\ref{sec:impl}. But all we
need is strong consistency with conditional writes.

The last-to-finish egress invokes the sink function with a vector of pointers
to each upstream function's stored output. The pointers are the in same order
as the vector of upstream function names. The ingress on the sink function
dereferences each point by reading from the data store and passes a vector of
output values to its user function.

Fan-in is a critical control-flow pattern to support and distinguishes \name{}
from ad-hoc trigger-based composition. \dhl{Do we want to constrast here? and
what should we say? Different from ad-hoc compositions: 1. use of data store
2. Control flow logic not only on the caller but also on the callee 3.
standard, off-the-shelf primitives that's not application specific that
developers build from scratch.}


\dhl{Propose name change for gadget -> nexus}

% chain = invoke, fan-out = a bunch of invoke, one for each continuation;
% Additionally, in the case of Map, one invoke for each element of the array
% (output of the user function).  fan-in .... well.... let's see. The semantics
% is: invoke the fan-in function once when all of its inputs are ready. In
% practice, it is each branch synchronize over the data store to see if all
% branches have completed. The last branch to complete calls invoke on the fan-in
% function, and pass it pointers to all branches results/checkpoints in the data
% store. The fan-in function first reads the branches' results, in order, via the
% pointers, then pass them as input to the user code.

\section{\name~IR}\label{sec:ir}

exporess control-flow logic. To express control-flow logic the IR first given
each function instance a unique name.

The naming scheme needs to encompass dynamic patterns such a map.

\begin{itemize}

    \item intermediate representation language that primarily encodes gadgets

    \item In practice, the \name{} IR is a set of configuration files (by default named
    \texttt{unum\_config.json}), one for each function node in the workflow.
    Every function has a \texttt{unum\_config.json} file deployed with it.

    \item Gadgets are encoded in the \texttt{"Next"} field. Give example for
    each gadget with code snippets. Explain the \texttt{"Name"} and
    \texttt{"InputType"} field.

    \item With the fan-in gadget, and the \texttt{"Values"} field, explain the
    need for IR to give each function node a unique name. The naming scheme
    works for dynamic patterns such as \texttt{map} where the number of
    branches is unknown at compile time. Explain the naming scheme.

    \item Fan-in expresses data dependencies. More flexible/expressive than
    Step Functions because SF only supports dependencies within a state. unum
    fan-in can specify any function in the workflow.

    \item The IR language primarily encodes gadgets but also provides other programmable
    constructs that enables dynamic behavior at runtime. For example
    Conditional: branch, termination condition for a loop, foward condition
    for a selective set of parallel branches. Value names in fan-in can
    include glob patterns to support fan-in of varying size. unum expands glob
    patterns during runtime. "Next Payload Modifiers" with runtime variables.
    Have a table of programmable constructs? (Conditional, \$ret, \$0, \$size)

\end{itemize}




\section{Runtime}

\begin{itemize}

    \item (focus on how gadgets are implemented and leave execution guarantee
    to its own section)

    \item \name{}'s runtime transparently interposes on user code's entry and
    exit such that developers do not change how they write application code at all.

    \item \name{} runtime implements all gadgets. And execute the assigned
    gadgets and control-flow logics based on the assigned IR file.

    \item Runtime has a particular input format in JSON.
    Figure~\ref{fig:input-format} shows the input format.

    \item \texttt{"Data"} field contains the input data to the user code. If
    the data is directly sent in the function invocation via HTTP (i.e.,
    \texttt{"Source": "http"}), the value of the "Value" field is passed to
    user code unchanged. If the \texttt{"Source"} field shows that the data is
    in the intermediary data store (e.g., \texttt{"Source": "dynamodb"}),
    \name{} reads the data from the data store and then pass it to user code.

    \item In practice, \name{} only passes data via the data store in the case
    of fan-in where results of multiple functions are needed. In that case,
    the \texttt{"Value"} field is an array of names of function checkpoints.
    \name{} reads each checkpoint in the array in order and passes the data as
    an array to user code.

    \item \texttt{"Session"} field contains a UUID string that uniquely
    identifies a workflow invocation. \name{} runtime on the entry function
    creates the UUID string when the function is invoked. The string
    propagates to all downstreams function instances that are part of the
    workflow invocation. Entry function can be any function on the graph. You
    can start with any function on the graph and invoke only a subsection. As
    long as the \texttt{"Session"} does not exist, \name{} runtime will create
    one.

    \item Checkpoint names are prefixed with the \texttt{"Session"} UUID
    string so that instances of the same function from different invocations
    do not overwrite each others' results.

    \item "Fan-out" field are part of fan-out functions' input payload.
    \name{} assigns each fan-out function an index.

    \item \name{} supports nested fan-outs with the \texttt{"OuterLoop"} field
    that is recursive.

    \item The "Index" in the "Fan-out" implements the branch name of the
    naming scheme in
    \S\ref{sec:ir}.

\end{itemize}

\section{Execution Guarantees}

Challenges to providing strong execution guarantees:

\begin{enumerate}

    \item Functions in the workflow can fail at any point.

    \item FaaS engines only provide at-least-once execution guarantee on
    individual functions. Triggering a function once might result in duplicate
    executions.

\end{enumerate}

Given the limitation of the underlying FaaS engines, \name{} cannot guarantee
exactly-once execution on \emph{individual functions}. However, \name{}
guarantees:

\begin{enumerate}

    \item At-least-once invocation on individual functions. This ensures that a
    workflow invocation will not just stuck somewhere and not proceed.

    \item In a particular workflow invocation, a particular function will
    always be invoked with the exact same input.

    \item Each workflow invocation will produce exactly one result. Even if
    there happen to be duplicate executions of functions, and even if the
    functions are non-deterministic, only a single result is recorded, while
    any duplicate and potentially diverging results are discarded.

\end{enumerate}

When a workflow execution crashes before finishing, \name{} only retries the
function where the crash happens instead of restarting the entire workflow.
\name{} leverages the automatic retry feature that most FaaS engines already provide
for individual functions~\cite{azure-functions-retry,
google-cloud-functions-retry, aws-lambda-async-invoke}.

\name{} checkpointing process is the following:

First, before running user code, the ingress checks to see if a checkpoint
already exists for the invocation. If it does, \name{} skips running user code
and uses the checkpoint's content as the result. The egress will invoke the
continuation again in case the prior execution crashed after checkpointing but
before running the continuation. This makes sure that if a step in the
workflow has completed and persisted, it will not run again.

If a checkpoint doesn't exist, unum runs the user code. After the user
function completes, unum checkpoints the result into the data store and runs
the continuation.


\section{Old Design}\label{sec:design}

% Notes(alevy):
%
% * Consider splitting design section into separate sections for System Architecture and IR/Language
% * Design needs to describe the (non)-architecture of unum. There are pieces of functionality that typically exist in a controller that still exist in unum, they are just embedded in
% unum runtime. Maybe we can give this 'component' a name, like a distributed controller (the unum runtime then implements the distributed controller).
% * The design includes too many implementation specific details, and doesn't really describe the high level operations. If the big question is 'how do you implement fan-out / fan-in
%   in a distributed way?', the section should answer that directly (same for other interactions which are much simpler).
% * Similarly, we need to describe the interactions that unum provides and sort of hand-wave argue that these are the right set of interactions (perhaps by pseudo-code example).
%
% Terms to use:
% * 'Gadgets': The interactions that are encoded in unum_config (fan-out, fan-in, chain, etc)
% * 'Execution graph': where nodes are either user-code (Lambda functions) or gadgets, edges exiting user-code always enter gadgets. Instead of talking about
%    continuations (I really don't think that Unum has something that looks like continuations), this execution graph is the IR and the runtime reifies the
%    execution graph by embedding gadgets in lambdas themselves.

%\begin{figure*}[t]
%    \centering
%    \includegraphics[width=\textwidth]{figures/unum-arch.pdf}
%    \caption{\name{}'s architecture}
%    \label{fig:arch}
%\end{figure*}

\begin{figure}[]
    \begin{minted}[
    frame=single,
    linenos,
    fontsize=\footnotesize
  ]{yaml}
Globals:
  ApplicationName: unum-iot-chain
  UnumIntermediaryDataStoreType: dynamodb
  UnumIntermediaryDataStoreName: iot-intermediary
  Checkpoint: true
  Debug: false
Functions:
  Aggregator:
    Properties:
      CodeUri: aggregator/
      Runtime: python3.8
      Start: true
  HvacController:
    Properties:
      CodeUri: hvac_controller/
      Runtime: python3.8
      Policies:
        - AmazonSQSFullAccess
  \end{minted}
    \caption{The unum template for an IoT HVAC controller
    application. It lists all resources of the workflow (Two functions:
    \texttt{Aggregator} and \texttt{HvacController}. A DynamoDB table
    \texttt{iot-intermediary} used as the intermediary data store) and
    specifies workflow-wide options such as checkpoint, debug and
    application name.}
    \label{fig:iot-unum-template}
\end{figure}


\begin{figure}[]
    \begin{minted}[
    frame=single,
    linenos,
    fontsize=\footnotesize
  ]{json}
{
  "StartAt": "Aggregator",
  "States": {
    "Aggregator": {
      "Type": "Task",
      "Resource": "Aggregator",
      "Next": "HvacController"
    },
    "HvacController": {
      "Type": "Task",
      "Resource": "HvacController",
      "End": true
    }
  }
}
    \end{minted}
    \caption{A Step Functions state machine for an IoT HVAC controller
    application. The workflow is a chain of two functions \texttt{Aggregator}
    and \texttt{HvacController}. When \texttt{Aggregator},
    \texttt{HvacController} should be invoked with \texttt{Aggregator}'s
    result. Note that normally the \texttt{"Resource"} fields are the deployed
    Lambda's ARN. Here they are function names defined in the unum template
    (see Figure~\ref{fig:iot-unum-template})}
    \label{fig:iot-sf}
\end{figure}

\begin{figure}[]
    \begin{minted}[
    frame=single,
    linenos,
    fontsize=\footnotesize
  ]{json}
{
    "Name": "string",
    "Next": {
        "Name": "FunctionName",
        "InputType": "Scalar" | "Map" | "Fan-in",
        "Condtional": "BooleanExpression"
    },
    "Next Payload Modifiers" : ["str"],
    "Checkpoint": "True | False",
    "Start": "True | False"
}
    \end{minted}
    \caption{}
    \label{fig:unum-config-lang}
\end{figure}

\begin{figure}[]
    \begin{minted}[
    frame=single,
    linenos,
    fontsize=\footnotesize
  ]{json}
{
    "Data": {
        "Source": "http | <data store type>",
        "Value": "<JSON object> | [<data store pointers>]"
    },
    "Session": "uuid",
	"Fan-out": {
        "Index": "int",
        "Size": "int",
        "OuterLoop": {
            "Index": "int",
            "Size": "int"
        }
    }
}
    \end{minted}
    \caption{}
    \label{fig:input-format}
\end{figure}

A key feature in \name{}'s approach is that it is designed as a runtime on top
of the serverless abstraction without requiring any supplemental services
added to current infrastructures. Workflows built with \name{} can execute in
any environment that has a FaaS engine with an asynchronous invocation API and
a named data store that supports strongly consistent reads. Both are readily
available on all existing serverless platforms.

Architecturally, \name{} eliminates the need for any specialized, hosted
processes that execute workflows, manage system states or broker
inter-function communication. FaaS functions are the only compute entity in
\name{} that runs code. They execute both user code and the \name{} runtime. The
\name{} runtime directly invokes the next function in the workflow after user
code completes and leverages a checkpointing mechanism to ensure strong
execution guarantees.

We continue this section by first giving an overview of unum's architecture
(\S~\ref{sec:design-overview}) and outlining the serverless abstractions and
services that \name{} builds upon (\S~\ref{sec:design-req}). We then
describe the programming interface that \name{} supports
(\S~\ref{sec:design-interface}). Next, we discuss the \name{} IR
(\S~\ref{sec:design-ir}) which expresses serverless workflows using
continuations. Finally, we describe \name{}'s runtime
(\S~\ref{sec:design-runtime}) and execution guarantees
(\S~\ref{sec:design-exec-gntee}).

\subsection{Overview}\label{sec:design-overview}

Figure~\ref{fig:arch} illustrates \name{}'s high-level architecture. At
compile-time, a frontend compiler transforms workflows written for existing
systems (e.g., an AWS Step Functions state machine that orchestrates Lambdas)
into a continuation-based, platform-independent intermediary representation.
In practice, the IR is a set of configuration files
(\texttt{unum\_config.json}), one for each function in the workflow, written
in the \name{} configuration language (\S~\ref{sec:design-config-lang}).

Next, a backend compiler compiles the IR into platform-specific packages that
are executable in a particular target environment (e.g., AWS Lambda with
DynamoDB as the data store). Deploying the packages will create a set of FaaS
functions (e.g., Lambdas) and an intermediary data store (e.g., a DynamoDB
table or S3 bucket).

A workflow is invoked by triggering the entry function. Each unum workflow has
one and only one entry function. \shadi{does it have to be one entry point? why?}

At execution time, each function runs both the user code and the unum runtime.
The unum runtime manages the function's checkpoint and invokes the
continuation asynchronously after user code completes. unum uses checkpoints
to provide strong execution guarantee. Every function invocation in unum is
assigned a unique name and the runtime uses the name for the instance's
checkpoint in the data store.

\subsection{System Requirements}\label{sec:design-req}

\name{} utilizes two services that are readily available on all existing
serverless platforms:

\begin{enumerate}

	\item A Function-as-a-Service engine that supports an
	\emph{asynchronous invocation API}.

	\item A shared data store that supports creating and reading items by
	 their unique names.

\end{enumerate}

\subsubsection{FaaS engine with asynchronous invocation}

Support for asynchronous invocation is universal across all major FaaS
engines, including AWS Lambda~\cite{aws-lambda-async-invoke}, Azure
Functions~\cite{azure-functions-async-invoke}, Google Cloud
Functions~\cite{google-cloud-functions-async-invoke}, and popular open-source
options such as OpenWhisk~\cite{openwhisk-async-invoke} and
OpenFaaS~\cite{openfaas-async-invoke}.

\name{} chooses asynchronous invocation because functions in \name{} directly invoke
their immediate downstream functions and asychronous invocation avoids
timeouts, and double billing. If only synchronous invocation is available, a
function has to idly wait for all of its downstream functions, immediate or
not, to complete, risking function timeouts and incurring double billing.

\paragraph{At-least-once invocation}

FaaS engines normally only guarantee at-least-once invocation of
asynchronously triggered
functions~\cite{google-cloud-functions-exec-guarantee,
aws-lambda-async-invoke, azure-functions-exec-guarantee}. A single invocation
might result in the FaaS engine running the same function more than once. Such
duplicate invocations are especially problematic when the function is
non-deterministic (i.e., given the same input, the function might produce
different output across runs) and all serverless providers simply urge
developers to write deterministic functions to avoid incorrect behavior.

From a workflow system's perspective, it has a few options to work with an
at-least-once FaaS engine: (1). it can equire all functions to be
deterministic (2). make sure functions are invoked exactly-once (3). ensure
that even if a non-deterministic function executes more than once, only one of
the results is taken as the final result and propagated to the rest of the
workflow.

\name{} takes the third approach as the first is restrictive and the second is
unattainable. \name{} uses a checkpointing mechanism to make sure only the
execution that finishes first is taken as the final result and propagated
downstream. Duplicates are simply discarded. We discuss the details in
\S~\ref{sec:design-runtime}.

\subsubsection{Named data store}

Functions in \name{} workflows use a named data store to store checkpoints and
other intermediary data (\S~\ref{sec:design-runtime}) during execution. The
data store needs to support creating and later retrieving objects with unique
names.

There is a wide variety of data store services that can support \name{}
workflows, including object storage (e.g., Amazon S3, AZure Blob Storage),
NoSQL databases (e.g., DynamoDB, Cosmos DB) and key-value stores (e.g.,
Redis).

Nearly all of the storage services above have a "serverless" option that
requires no explicit provisioning and users only pay for what they use.

\paragraph{Consistency Requirements}

\name{} requires the data store to be strongly consistent.

Strongly consistent reads (read that return the most up-to-date data,
reflecting all prior writes that were successful) are important for the
correctness of aggregations (e.g., fan-in), because it prevents the
possibility where all upstream functions have completed and written their
checkpoint but none of them sees that all have completed and thus never invoke
the fan-in function.

Note that strong consistency alone is not enough for correct execution of
aggregation patterns because multiple functions might detect that all fan-out
branches have completed and thus invoking the fan-in functions more than once.
\name{} runtime contains additional synchronization logic to make sure the fan-in
function is only invoked once. We discuss the details in
\S~\ref{sec:design-runtime}


\subsection{Programming Interface}\label{sec:design-interface}

\name{} does not introduce a new programming interface for building serverless
workflows. Developers can use familiar frontend languages from existing
systems, such as the Amazon State Language for AWS Step Functions.

Moreover, \name{} lets developers write component functions in
workflows exactly the same way as they would for regular functions. In fact,
orchestration-related logic are entirely transparent from user functions'
perspective. There is no additional libraries that user code needs to import
or use.

Figure~\ref{fig:arch} gives an example of using Step Functions state machine
to define the workflow that orchestrates functions in Python. Each
function is defined in its own directory with the exact same content as you
would have for regular functions that are not part of any workflow.

The \emph{unum template} is a YAML file that lists all the resources in the
workflow and specifies workflow-wide configurations.
Figure~\ref{fig:iot-unum-template} gives an example of an IoT HVAC controller
application's template.

The frontend compiler transforms the workflow definition to a
continuation-based, platform-independent intermediary representation (IR)
which we discuss in the next section.



\subsection{\name{} IR}\label{sec:design-ir}

The \name{} intermediary representation expresses FaaS workflows using
continuations. A continuation defines (1). which function(s) to invoke (2).
what input to invoke it with.

After running user code, the unum runtime invokes the continuation with the
user code's result as the input. Every function in a unum workflow has 0, 1 or
more continuations. The set of continuations from all functions form the
complete workflow.

unum continuations form graphs with functions as vertices. They support common
orchestration patterns such as chaining, branching, fan-out and fan-in. unum
also supports \texttt{for} or \texttt{while} loop (fold) with continuation
graphs that have directed cycles.

A key benefit of the continuation-based IR is to distribute workflows'
orchestration logic to individual component functions so that it can execute
without a centralized orchestrator. In an orchestrator-based workflow system,
functions will need to call back to the orchestrator service to signal
completion and it is the orchestrator who will then invoke the next function.

In contrast, unum assigns each function a set of continuations \emph{at
compile-time}. After user code completes, each function directly invokes its
continuations asynchronously, without involving any supplemental services.

% \subsubsection{Naming scheme}

% \name{} assigns each function instance a unique name. The name consists of the
% function's name and, if the instance is part of a fan-out, the instance's
% fan-out index.



\subsubsection{\name{} configuration language}\label{sec:design-config-lang}

In practice, the \name{} IR is a set of configuration files (by default named
\texttt{unum\_config.json}), one for each function in the workflow, written in
the \name{} configuration language.

Figure~\ref{fig:patterns} shows 2 examples patterns and how to express them in
the \name{} IR.

The configuration language provides constructs to define continuations. Each
continuation is a JSON object. It specifies the name of the function to invoke
(the \texttt{"Name"} field), what its inputs are (the \texttt{"InputType"}
field) and an invoke condition (the \texttt{"Conditional"} field).

The function name needs to match one of the names in the \name{} template. The 

Continuations are defined in the \texttt{"Next"} field. The value is
either a single JSON object if there is only one continuation or an array
of JSON objects if there are multiple continuations.

% Figure~\ref{fig:iot-sf} shows a Step Functions state machine for an IoT HVAC
% controller workflow which is a chain of two functions. Figure~\ref{fig:iot-ir}
% shows the generated IR in the form of two \texttt{unum\_config.json} files.

\begin{itemize}

	\item Every function has a \texttt{unum\_config.json} file deployed with
	it.

	\item Continuations are defined in the \texttt{"Next"} field. The value is
	either a single JSON object if there is only one continuation or an array
	of JSON objects if there are multiple continuations.

	\item Each continuation is a JSON object. It specifies the name of the
	function to invoke next in the \texttt{"Name"} field, what its inputs are
	in the \texttt{"InputType"} field and an invoke condition in the
	"Conditional" field.

	\item \texttt{"InputType"} has three possible values: \texttt{"Scalar"},
	\texttt{"Map"} and \texttt{"Fan-in"}.

	\item \texttt{"Scalar"} is used when the only input to the next function
	is the output of this function.

	\item \texttt{"Scalar"} supports chaining, where the input to the next
	function is the output of the previous function, and fan-out, where
	multiple functions are triggered in parallel with the output of the
	previous function.

	\item Give example \texttt{unum\_config.json} files of chaining and
	fan-out with \texttt{"Scalar"}?

	\item \texttt{"Map"} supports a different form of fan-out where the output
	of this function is an array and for each element of the array, the
	runtime invokes one instance of the next function with the element as
	input.

	\item \texttt{"Fan-in"} is used for aggregations when the next function to
	invoke depends not only on the result of this function but also on those
	of other functions. (Give a concrete example here?)

	\item Values that the next function depends on are listed in the
	\texttt{"Values"} field. Each value is identified by the unique name of
	the function instance that produces it. We discuss \name{}'s naming scheme
	in the next section.

	\item Value names can include glob patterns to support fan-in of varying
	size. unum expands glob patterns during runtime.

	\item The "Conditional" field is a boolean expression. Only when the
	boolean expression evaluate to Ture would the continuation be invoked.

	\item Workflow entry functions have \texttt{"Start: true"}.

	\item "Next Payload Modifiers" is an interface to modify \name{}'s runtime
	metadata. 

\end{itemize}


\subsection{Runtime}\label{sec:design-runtime}

\begin{itemize}

	\item \name{}'s runtime interposes on user code's entry and exit (See
	Figure~\ref{fig:arch}. On ingress, \name{} runtime reads the input data
	and then pass it to user code. When user code returns, the egress takes
	the return value, write it to the data store as a checkpoint and invokes
	the continuation.

	\item Inputs to unum functions are JSON objects.
	Figure~\ref{fig:input-format} shows the input format.

	\item \texttt{"Data"} field contains the input data to the user code. If
	the data is directly sent in the function invocation via HTTP (i.e.,
	\texttt{"Source": "http"}), the value of the "Value" field is passed to
	user code unchanged. If the \texttt{"Source"} field shows that the data is
	in the intermediary data store (e.g., \texttt{"Source": "dynamodb"}),
	\name{} reads the data from the data store and then pass it to user code.

	\item In practice, \name{} only passes data via the data store in the case
	of fan-in where results of multiple functions are needed. In that case,
	the \texttt{"Value"} field is an array of names of function checkpoints.
	\name{} reads each checkpoint in the array in order and passes the data as
	an array to user code.

	\item User code is oblivious of the \name{} runtime. \name{} does not
	change how user code is written or its interface.

	\item \texttt{"Session"} field contains a UUID string that uniquely
	identifies a workflow invocation. \name{} runtime on the entry function
	creates the UUID string when the function is invoked. The string
	propagates to all downstreams function instances that are part of the
	workflow invocation.

	\item Checkpoint names are prefixed with the \texttt{"Session"} UUID
	string so that instances of the same function from different invocations
	do not overwrite each others' results.

	\item "Fan-out" field are part of fan-out functions' input payload.
	\name{} assigns each fan-out function an index.

	\item \name{} supports nested fan-outs with the \texttt{"OuterLoop"} field
	that is recursive.

	\item Each function instance is assigned a unique name. The name consists
	of the function's name and, if the function is part of a fan-out, its
	fan-out indexes. Example?

	\item Fan-in semantics. Only invoke the continuation once when all inputs
	are ready. For data stores with atomic data structures, we use them to
	synchronize. For data stores without atomic data structures, \name{}
	pre-select that last function in the \texttt{"Value"} array to wait for
	all other functions to complete.

\end{itemize}

% \subsubsection{Checkpoints?}


% check checkpoint existence before running user function

% run user function or read from existing checkpoint

% write checkpoint or pass

% run continuation

% checkpoints are used both for guaranting execution correctness and for
% aggregation patterns such as fan-in.


\subsection{Fault tolerance and execution guarantee}\label{sec:design-exec-gntee}

Challenges to providing strong execution guarantees:

\begin{enumerate}

	\item Functions in the workflow can fail at any point.

	\item FaaS engines only provide at-least-once execution guarantee on
	individual functions. Triggering a function once might result in duplicate
	executions.

\end{enumerate}

Given the limitation of the underlying FaaS engines, \name{} cannot guarantee
exactly-once execution on \emph{individual functions}. However, \name{}
guarantees:

\begin{enumerate}

	\item At-least-once invocation on individual functions. This ensures that a
	workflow invocation will not just stuck somewhere and not proceed.

	\item In a particular workflow invocation, a particular function will
	always be invoked with the exact same input.

	\item Each workflow invocation will produce exactly one result. Even if
	there happen to be duplicate executions of functions, and even if the
	functions are non-deterministic, only a single result is recorded, while
	any duplicate and potentially diverging results are discarded.

\end{enumerate}

From users' perspective, if none of the functions in the workflow have
external side-effects (e.g., writing to external services), the workflow will
appear to execute exactly-once.

\name{} leverages the at-least-once execution guarantee of FaaS engines to
ensure that functions run at least once.

When a workflow execution crashes before finishing, \name{} only retries the
function where the crash happens instead of restarting the entire workflow.
\name{} leverages the automatic retry feature that most FaaS engines already provide
for individual functions~\cite{azure-functions-retry,
google-cloud-functions-retry, aws-lambda-async-invoke}.

\name{} uses checkpointing to make sure that even if a function executes more
than once, whether due to retries or duplicate invocations, only the result of
one of the executions is taken as the final result, and only the final result
is propagated downstream to the rest of the workflow.

\name{} checkpointing process is the following:

First, before running user code, the ingress checks to see if a checkpoint
already exists for the invocation. If it does, \name{} skips running user code
and uses the checkpoint's content as the result. The egress will invoke the
continuation again in case the prior execution crashed after checkpointing but
before running the continuation. This makes sure that if a step in the
workflow has completed and persisted, it will not run again.

If a checkpoint doesn't exist, unum runs the user code. After the user
function completes, unum checkpoints the result into the data store and runs
the continuation.

\section{Implementation}\label{sec:impl}

We implement a prototype \name{} runtime that supports AWS Lambda and Google
Cloud Functions, leveraging DynamoDB and Firestore, respectively, as
intermediate datastores. We also implemented a front-end compiler that takes
arbitrary AWS Step Function state-machine definitions and compiles them to
\name{} IR, which can be run on either. Currently our runtime only supports
Python functions and is itself written in 1,119 lines-of-code. The Step
Functions compiler is 549 lines-of-code.

Implementing the runtime primarily requires specializing high-level
functionality the IR depends on for a particular FaaS platform and datastore.
The FaaS platform must support asynchronous function invocation and the
datastore must be linearizable with support for atomic creation and set
operations.

Importantly, we choose datastores and primitives that only incur per-use costs,
rather than including baseline provisioning cost, and scale on-demand. For
example, we use DynamoDB in on-demand capacity mode, rather than provisioned
capacity mode, and avoid long-running services such as a hosted Redis or cache.
As a result, \name{} workflows, like their centralized counterparts, incur
resource costs \emph{only} to actually execute a workflow.

\subsection{AWS Lambda \& DynamoDB}

Asynchronous invocation in Lambda is natively supported. In particular, the
Lambda \texttt{Invoke} API is asynchronous when passed \texttt{InvocationType=Event}.

DynamoDB organizes data into tables, with each item in a table named by a key.
Within tables, items are unstructured by default. Our implementation of \name{}
uses a single table for each workflow invocation. Each item in the table
corresponds to a checkpoint, or a coordination set for fan-in or garbage
collection.

DynamoDB supports atomic item creation by passing the conditional flag
\texttt{attribute\_not\_exists} to the \texttt{put\_item} API call. We use this
for creating both checkpoint blobs and coordination sets. DynamoDB supports set
addition natively using the \texttt{Map} field type. In particular, we use
update expressions to atomically set a named map element to true.  As an
optimization, we use the \texttt{ALL\_NEW} flag when adding to a set to
atomically get the new value after a set in a single operation.

\subsection{Google Cloud Functions \& Firestore}

Google Cloud Functions (GCF) do not have an asynchronous invocation API.
Instead, we allocate function-specific publish-subscribe queues and subscribe
each function to its respective queue.  \name{} then performs asynchronous
invocation of a function by publishing the input data as an event to the
function's queue.

Firestore organizes data into logical collections (which are created and deleted
implicitly) containing unstructured items, named by a unique key. Similar to
DynamoDB, we use a separate collection for each workflow. Atomic item creation
is supported using a special \texttt{create} API call, which only succeeds if
the key does not already exist. Firestore supports an \texttt{Array} field type
which can act as a set by using the \texttt{ArrayUnion} and \texttt{update}
operation, which atomically sets the field to the union of its existing elements
and the provided elements. The \texttt{update} operation always returns the
new value data.


\section{Evaluation}\label{sec:eval}

We evaluate \name{} along 3 metrics: (1). cost, (2). latency, and (3).
expressiveness, to answer the following questions:

\begin{enumerate}

    \item What is the cost of running applications on \name{} and how does it
    compare to existing serverless workflow systems? Specifically, how much
    additional costs does the \name{} runtime incur in Lambda duration
    billing? And how much does it cost to write and read checkpoints to and
    from the intermediary data store?

    \item What is the latency performance of representative applications on
    \name{}? And how does it compare to existing serverless workflow systems?

    \item How expressive is \name{}'s representation of serverless workflows
    (the \name{} IR)? Can one build complex, real-world applications with
    \name{}?

\end{enumerate}


We use a suite of micro- and macro-benchmarks. The micro-benchmarks target the
basic operations of \name{} (e.g., invoking a continuation, checkpointing,
etc.) and building-block orchestration patterns (e.g., chaining, fan-out and
fan-in) to understand unum's performance characteristics and cost benefits.

The macro-benchmarks consists of 4 real-world applications, taken from
serverless repositories and prior research work, aiming to
evaluate unum's expressiveness and end-to-end performance and costs.

We show that 

\begin{itemize}

    \item over 97\% of \name{}'s latency overhead comes from API calls to
    Lambda and data stores, which means the bulk of \name{}'s performance will
    automatically improve with the underlying platform (e.g., a faster Lambda
    or data store) without any modification to \name{} itself.

    % \item The additional Lambda duration billing for executing \name{} runtime
    % is negligible across all data sizes

    \item \name{} is slightly faster (11-28\%) in chaining performance and
    much faster in parallel fan-out and fan-in performance (up to 4.58x),
    especially at higher level of parallelism, than Step Functions.

    \item \name{} delivers more than one order-of-magnitude cost savings for
    almost all applications we evaluated, even when using the more expensive
    DynamoDB as the intermediary data store. The applications we use cover all
    orchestration patterns that Step Functions currently support.

    \item \name{} is able to express all orchestration patterns that Step
    Functions currently support. Additionally, with the ExCamera
    implementation, we demonstrate that \name{} can express fold or for loops
    and support pipeline parallelism, neither of which is expressible in Step
    Functions.

\end{itemize}

\subsection{Experimental setup}

We run all experiments on AWS, region \texttt{us-west-1} and costs numbers
reflect \texttt{us-west-1} pricing. We configure lambdas to 128MB memory size
unless otherwise specified and use on-demand capacity mode for DynamoDB. To
avoid function cold starts, we pre-warm functions by running the workflow a
few doze times before collecting measurement.

% S3 buckets all have Versioning turned on.

We compare against Step Functions' \emph{Standard} Workflows as the baseline.
Similar to \name{}, the Standard Workflows persists execution states on every
state transition (i.e., completing one function and starting the next
function), and always returns exactly one response for one workflow
invocation~\cite{aws-step-functions-exec-gntee}.

We do not consider Step Functions' Express Workflows in our comparison because
of its weaker execution guarantee, namely the same invocation could result in
multiple, potentially different results if any part of your workflow logic is
nonidempotent~\cite{aws-step-functions-exec-gntee}.

Note that even though Step Functions claims that the Standard Workflows
provides "exactly-once workflow
execution"~\cite{aws-step-functions-exec-gntee}, it is not clear whether it
implies exactly-once execution for component functions of the workflows. Our
interpretation is that the internal states of a standard workflow will appear
to execute exactly once, but component functions might not run exactly-once
due to failures and retries, which is identical to \name{}.

\subsection{Microbenchmarks}

\begin{figure}[t!]
    \centering
    \includegraphics[width=\columnwidth]{figures/TotalAdditionalLatency.pdf}
    \caption{Total latency unum incurs for a single state transition. We use a
    chain of two functions (\texttt{F->G}) that simply return their input.
    \texttt{data size} is the output data size of \texttt{F}, which in turn is
    the amount of data \texttt{F} writes to checkpoint and the input data size
    of \texttt{G}.}
    \label{fig:totallatency}
\end{figure}

\begin{figure}[t!]
    \centering
    \includegraphics[width=\columnwidth]{figures/OpLatency.pdf}
    \caption{Average latency incured by unum primitives. The bottom right
    figure shows the total latency of the one Lambda invoke call and two
    storage accesses as a percentage of total runtime overhead.}
    \label{fig:oplatency}
\end{figure}

\begin{figure}[t!]
    \centering
    \includegraphics[width=\columnwidth]{figures/OpLatency-pct.pdf}
    \caption{Latency breakdown of unum runtime. The majority of the latency is
    from the Lambda invoke call and the two storage accesses.}
    \label{fig:oplatency-pct}
\end{figure}


\begin{figure}[t!]
    \centering
    \includegraphics[width=\columnwidth]{figures/TotalCost.pdf}
    \caption{Total costs comparison of 1 million state transitions between
    Step Functions and \name{}}
    \label{fig:total-costs}
\end{figure}

\begin{figure}[t!]
    \centering
    \includegraphics[width=\columnwidth]{figures/MapMicroLatency.pdf}
    \caption{End-to-end latency (logscale) of parallel fan-out and fan-in.
    Lower is better. unum scales much better than the current version of Step
    Functions (experiments ran on Nov. 15, 2021) for parallel fan-out
    workloads. The result is not arguing that orchestrator-based workflow
    systems fundamentally scales worse than unum. In fact, the main cause of
    high latency in Step Functions is throttling of concurrent iterations when
    the fan-out size exceeds 40~\cite{aws-step-functions-map-state}. Although
    Step Functions does not elaborate on the reason for such throttling, there
    is little reason to believe that the maximum allowable concurrency is
    fundamentally capped around 40 for parallel workloads. However, this
    result does demonstrate an important downside of adding supplemental
    hosted services to support new workloads: any services added must also
    perform well compared with the highly scalable FaaS substrate and
    application developers must work with any restrictions that the services
    impose. unum avoids the additional work of building and maintaining hosted
    orchestrators and directly leverages the scalability of Lambda to achieve
    better parallel performance.}
    \label{fig:mapmicrolatency}
\end{figure}

\begin{figure}[t!]
    \centering
    \includegraphics[width=\columnwidth]{figures/ChainMicroLatency.pdf}
    \caption{End-to-end latency of chaining N functions. Lower is
    better. Each function simply returns the input data without performing any
    computation. Results in the figure are for 1KB of input data.}
    \label{fig:chainmicrolatency}
\end{figure}


\paragraph{How much additional latency does unum incur? How much does that
translate to costs?}

We first look at the performance and costs of the unum runtime. We measure the
latency of each primitive and the total additional latency for unum to take a
checkpoint and invoke a continuation. We translate the additional latency to
dollar costs in Lambda duration and compare it against Step Functions' costs
for state transitions. Note that unum taking a single checkpoint and invoking
a single continuation is equivalent to one Step Function state transition.




\subsection{Cost comparison}

Total cost of a state transition:

\begin{itemize}

    \item For Step Functions, a state transition incurs a fixed cost of \$27.8
    per 1M state transitions, regardless of the data size.

    \item For unum, there are two parts to the cost:

    \begin{enumerate}

        \item Lambda Duration costs from running the \name{} runtime. That is
        the additional time it takes to execute \name{} runtime code. This
        cost depends on the memory size of the Lambda. We'll assume the
        largest available Lambda size of 3072 MB which costs \$0.05 per 1M ms.

        \item Costs of checkpointing. Here we are mainly concerned with the
        cost of the read and write operations. We are \emph{not} concerned
        with the cost of storing checkpoints because the intermediary data can
        be delete right after workflow execution.

    \end{enumerate}

    \item S3 charges \$5.5 per 1M PUT requests (for checkpoint write) and \$0.44
    per 1M GET requests (for checkpoint read).

    \item DynamoDB on-demand capcity mode charges reads and writes based on
    the data size. 1M requests for writing 1KB data costs \$1.3942 and 1M
    requests for reading 1KB data costs \$0.279. Note that in the case of
    DynamoDB, if no faults happen during execution, the checkpoint read will
    return "item not found", which costs the same as returning 1KB of data.

    \item For 1M state transitions, \name{}'s costs for S3:

    \[  r(d)\times0.05 + 5.94 \]

    and for DynamoDB:

    \[  r(d)\times0.05 + d\times1.3942 +0.279\],

    where $r$ is the total additional runtime of \name{}, $d$ is the data
    size.

\end{itemize}


\paragraph{How well does unum perform with orchestration primitives (chaining,
fan-out, fan-in)?}


Though we want to compare the performance of basic workflow operations (e.g.,
invoking a function, checkpointing) between unum and Step Functions, Step
Functions logs do not provide sufficiently granular timestamps for an
apple-to-apple comparison. Therefore, we instead measure the end-to-end
latency of buidling-block orchestration patterns -- chaining, fan-out and
fan-in. To make sure the measurements mostly reflect the performance of the
workflow system and not the latency of user code, we minimize function
duration to 1ms so that the majority of latency comes from the workflow
system.




\paragraph{What's the cost of checkpointing? The total cost of the equivalence
of a single Step Functions state transition.}




% \subsection{Key Results}

% Invocation latency and invocation latency vs input payload size.

% Invocation latency translate to price,, i.e., 1 million invocations = \$ vs Step Function each state transition.

% Checkpoint latency (including the first check if a checkpoint exists) and
% checkpoint latency vs input payload size up to the Lambda invocation payload
% size limit. Because if you write Lambda applications, once your data is larger
% than the input payload size, you'd do your own data management anyway. This is
% perhaps not ideal API but this is not something that unum changes. In fact,
% it's not clear whether we should change this because some types of data are
% better when used with a data store.

% Total, invocation latency + checkpoint latency vs input payload size and
% translate this to \$ and compare with Step Functions for for instance 1million
% invocations.

% Do the same for a fan-out and fan-in:

% The fan-out initiator is going to spend XX ms to invoke all fan-out functions.
% This number should be roughly linear to <fan-out size> * individual invocation
% latency. So the invocation latency for a single continuation should be
% sufficient.

% Each fan-out function needs to spend XX extra ms to write to DynamoDB and the
% last to finish is going to invoke the fan-in function.

% The fan-in function needs to spend XX extra ms to read from the intermediary
% data store.


% % -------------

% Chaining latency vs chain length, compare with Step Functions

% Map Latency vs map size, compare with Step Functions

%     Example of an artificial limitations on scalability with additional service approach

% % -------------

% real applications/macrobenchmarks or case study?

% Not sure how to present the results. All applications in the same table? or
% each application with its own subsection? What if we end up with some
% applications that deserve its own subsection and some don't? Answer: start by
% writing each application with its ownd subsection, pour everything out and
% then decide how to organize.



% % --------------

% (Not sure how to discuss the expressiveness advantage or whether we should
% discuss at all)

% Step Functions no way to express a for loop or fold.

% Step Functions no way to express pipeline parallelism.

% Step Functions fan-in needs to make sure the aggregate data doesn't exceed a
% limit, unum automatically passes data in using pointers to the intermediary
% data store.
\subsection{Macrobenchmarks}

\subsubsection{Applications}

% Non-contrived

% Representative

% Describe what each application does, What pattern they use, what 


% \begin{table}[]
% \begin{tabular}{llllll}
% \hline
%                      &                        & \multicolumn{2}{l}{\textbf{unum}}                                                                                                   & \multicolumn{2}{l}{\textbf{Step Functions}}                                                                     \\
% \textbf{Application} & Pattern                & \textit{e2e latency} & \textit{cost (per 1M exec.)}                                                                                 & \textit{e2e latency} & \textit{cost (per 1M exec.)}                                                             \\ \hline
% IoT Pipeline         & chain                  & 120.9ms              & $0.2*2+(73+28)*$0.0021+2*\$1.3942                                                                            & 226.52               & $0.2*2+ 4*$27.9                                                                          \\
% Text Processing      & fan-out, fan-in        & 562.69ms             & $0.2*6+ (105+149+70+68+144+100)*$0.0021 + 6*$1.3942+2*2*$0.279                                               & 552.46ms             & $0.2*5+7*$27.9                                                                           \\
% Wordcount            & chain, fan-out, fan-in & 410s                 & $0.2*(1+262+1+250+1) + (277+6264*262 + 348 + 667*250 +68)*$0.0021 +(1+262+1+250+1)*$1.3942 + 262*2*$0.279+ 250*2* $0.279    & 898s                 & $0.2*(1+262+1+250+1) + (5913*262 + 154 + 633*250 +5)*$0.0021 +(1+262+1+1+250+1+1)*\$27.9 \\
% ExCamera             & chain, fan-out, fold   & 84s                  & $0.2*(1+16+15+15+14) + (6500+1500+350+4500+5000)*$0.0021+ (1+16+15+15+14)*$1.3942 + 15*2*$0.279+14*2*\$0.279 & 98s                  & $0.2*(16+16+1+16+15)+(6300+1400+2+5500+5300)*$0.0021+(1+16+16+1+1+16+1+1)*\$27.9      \\ \hline
% \end{tabular}
% \end{table}

\begin{table*}[t]
\begin{tabular}{llllll}
\hline
                     &                        & \multicolumn{2}{c}{\textbf{unum}}                                                                                                   & \multicolumn{2}{c}{\textbf{Step Functions}}                                                                     \\
\textbf{Application} & \textbf{Pattern}                & \textit{e2e latency} & \textit{cost (per 1M exec.)}                                                                                 & \textit{e2e latency} & \textit{cost (per 1M exec.)}                                                             \\ \hline
IoT Pipeline         & chain                  & 120.9ms              & \$3.4005                                                                            & 226.52               & \$112                                                                          \\
Text Processing      & fan-out, fan-in        & 562.69ms             & \$12.0168                                               & 552.46ms             & \$196.3                                                                           \\
Wordcount            & chain, fan-out, fan-in & 410s                 & \$4904.79   & 898s                 & \$18113 \\
ExCamera             & chain, fan-out, fold   & 84s                  & \$150.91 & 98s                  & \$1530      \\ \hline
\end{tabular}
\end{table*}

\subsection{Writing Notes}

\subsubsection{What does success look like?}

\begin{enumerate}

    \item Expressiveness. That you can build a wide range of realistic
     applications with unum.

    \item Latency performance. That unum is on par with Step Functions.

    \item Cost. That running applications on unum is cheaper than Step
     Functions.

\end{enumerate}

We can tune the phrasing based on how much we promise in the Intro, but the
main metrics are the above three.

\subsubsection{Questions}

\begin{enumerate}

    \item How do we evaluate and present execution guarantee? Anything to show
     to convince our reader that it's correct?

    \item How do we evaluate other benefits that stem from a simpler design
     (the fact that unum gets rid of the needs for a separate orchestrator
     service), such as resource management, required staffing and other
     hosting costs? Further on the resource utilization point, do we want to
     say that dollar costs of running applications is a reasonable enough
     proxy to resource consumption and therefore lower price = less resource
     consumption = better resource utilization?

    \item Should we run experiments with S3 and present those numbers?

\end{enumerate}


% \section{Limitations \& Future Work}\label{sec:limitations}

\paragraph{Unsupported applications.} \name{} supports a superset of
applications that can be expressed using Step Functions, but there are
applications that do not fit \name{}'s constraints. In particular, \name{} only
supports statically defined control structures. For example, Durable Functions
expresses workflows dynamically as code and allows the developer to use
arbitrary logic to determine what the next workflow step should be at runtime.
This is not currently possible with \name{}.

\paragraph{Measurement error.} Due to the opaque design, implementation and
pricing of production workflow systems, such as Step Functions, comparisons in
our evaluations are limited in their explanatory power. In particular, we use
the current \emph{price} of Lambda, DynamoDB, and Step Functions as a proxy for
the \emph{cost} of providing these services. Of course, different platform
providers may choose different pricing schemes and prices may be either lower or
higher for a particular service than the underlying cost.

Similarly, performance measurements of Step Functions is done as a black box and
thus cannot determine which components of performance are due to design choices
(such as requiring additional network communication) and which are due to
deployment variables (queue depth, provisioned resources, etc).

\paragraph{Support for more languages and platforms.} While \name{} is designed
to run on any serverless platform that meets our minimal criteria, our current
implementation is only complete for AWS Lambda using DynamoDB or S3 for storage.
We are working on other backends, including for Azure Functions and Google
Functions, allowing the same applications to run on multiple platforms.

Moreover, we only described a front-end compiler that targets the Step Functions
description language. We intend to implement compilers for other workflow
description languages as well.

\section{Conclusion}\label{sec:conclusion}

Serverless platforms allow developers to construct applications from modular
programming units that can scale quickly and independently, promising
burst-scalability and fine-grained billing. Workflow orchestrators make building
complex applications out of these event-driven, asynchronous functions
reasonable, but may introduce performance, cost, scalability, and deployment
overhead to developers and platform providers. We designed and implemented
\name{}, a \emph{decentralized} workflow orchestrator that requires no
additional infrastructure to deploy, imposes no additional limits on
scalability, and performs as well as or better than centralized solutions while
providing similar expressiveness and execution guarantees.

\section{Related Work}\label{sec:related}

Overall characteristics of orchestrator based engines: The orchestrator
performs function invocations; functions send results back to correct the
orchestrator instance.

Why does this characterstics matter?

Questions:

\begin{itemize}
	\item Is it pure serverless? What does pure serverless mean?
	\item Any performance comparisons necessary?
\end{itemize}

\paragraph{gg}~\cite{gg-atc}

\begin{itemize}
	\item Target applications are what they call "everyday tasks", including code compilation, unit testing, video encoding and object recognition
	\item A thunk = a x86-64 executable and its input data

	A thunk is deterministic. Forcing a thunk multiple times will generate the same object.

	Data objects are named (naming scheme 3.1.1). Reading an object might involve forcing a thunk to generate the object. A thunk's input can be references to other thunk's outputs. gg uses this mechanism to express computation graphs.

	Forcing a thunk multiple time will always produce the same object with the same name. To run the same function with different input needs to define a different thunk (presumably a gg compiler can help simplify the step of defining thunks that have the same executable and different input. But the paper wasn't entirely clear on this and the source code shows application-specific compilation with limited programmability.).


	duplicate functionalities that Lambda already provides: job scheduling, retry, timeouts.



\end{itemize}


\paragraph{Triggerflow}~\cite{triggerflow}
See Section 4. implementation. Each workflow has its own workflow worker which is really a kubernetes pod. The workflow works process events and trigger actual FaaS functions.


\paragraph{Kappa}~\cite{Kappa}
Tasks communicate with the coordinator through
the remote procedure calls (RPCs) summarized in Table 2


\paragraph{Cloudburst}~\cite{cloudburst}
1. Orthogonal: They focus on building a data store with specialized APIs and consistency models for faas workflow
2. Differennt programming interface (TODO: more details)
3. Not built on real faas platform; can't run on aws, azure or google cloud


\paragraph{mu and Excamera}~\cite{excamera}

\paragraph{Boki}


\paragraph{PyWren}~\cite{pywren}



% Google Cloud Composer uses Airflow.
% https://github.com/apache/airflow/tree/main/airflow/example_dags
% https://cloud.google.com/composer/docs/samples

% Azure Durable Functions. https://docs.microsoft.com/en-us/learn/modules/create-long-running-serverless-workflow-with-durable-functions/

% Google Cloud Workflows.
% Example: https://codelabs.developers.google.com/codelabs/cloud-workflows-intro
% https://cloud.google.com/workflows
% Repository: https://cloud.google.com/workflows/docs/samples

 
% https://github.com/serverlessworkflow/specification/tree/main/examples



% https://github.com/kalevalp/hello-retail-baseline

%-------------------------------------------------------------------
\bibliographystyle{plain}
\bibliography{references}

%%%%%%%%%%%%%%%%%%%%%%%%%%%%%%%%%%%%%%%%%%%%%%%%%%%%%%%%%%%%%%%%%%%%%%%%%%%%%%%%
\end{document}
%%%%%%%%%%%%%%%%%%%%%%%%%%%%%%%%%%%%%%%%%%%%%%%%%%%%%%%%%%%%%%%%%%%%%%%%%%%%%%%%

%%  LocalWords:  endnotes includegraphics fread ptr nobj noindent
%%  LocalWords:  pdflatex acks
