\section{Discussion \& Limitations}\label{sec:limitations}

% \textbf{Access control and rate limiting.} Orchestrators often integrate
% with separte services to support access control. For instance, AWS IAM
% controls permissions to access Step Functions workflows. Individual function
% permissions within workflows are also controlled via IAM. \name{}
% applications can control access to workflows by setting the permission of
% entry functions. Rating limiting is another key functionality that
% orchestrators support either natively~\cite{aws-step-functions-quotas} or
% with separate frontend services~\cite{cloudfront-rate-limiting}. \name{}
% application can similarly control the rate of execution by setting the
% concurrency limits of entry functions in FaaS engines or in frontend
% services.

\textbf{Unsupported applications.} \name{} supports a superset of
applications that can be expressed using Step Functions, but there are
applications that do not fit \name{}'s constraints. In particular, \name{}
only supports statically defined control structures. For example, Durable
Functions expresses workflows dynamically as code and allows the developer to
use arbitrary logic to determine what the next workflow step should be at
runtime. This is not currently possible with \name{}.\par

\textbf{Measurement error.} Due to the opaque design, implementation and
pricing of production workflow systems, such as Step Functions, comparisons in
our evaluations are limited in their explanatory power. In particular, we use
the current \emph{price} of Lambda, DynamoDB, and Step Functions as a proxy
for the \emph{cost} of providing these services. Of course, prices may be
either lower or higher for a particular service than the underlying cost.

\textbf{Code Complexity.} While \name{} affords users more flexibility,
application-level orchestration increases code complexity for developers.
Coordination and exactly-once execution require careful design and
implementation to function correctly in a decentralized manner. Introducing
application-specific optimization also needs additional developer efforts than
using off-the-shelf patterns from provider-hosted orchestrators.

% \textbf{Support for more platforms.} While \name{} is designed to run on
% any serverless platform that meets our minimal criteria, our current
% implementation is only complete for AWS Lambda using DynamoDB or S3 for
% storage. We are working on other backends, including for Azure Functions and
% Google Functions.

\section{Conclusion}\label{sec:conclusion}

% Serverless platforms allow developers to construct applications from modular
% programming units that can scale quickly and independently, promising
% burst-scalability and fine-grained billing. Standalone orchestrators simplify
% building complex serverless applications but are inflexible and expensive to
% use. We designed and implemented \name {}, an application-level, decentralized
% orchestration system that runs as a library on unmodified serverless
% infrastructure without requiring additional services.

We designed and implemented \name {}, an application-level, decentralized
orchestration system that runs as a library on unmodified serverless
infrastructure without requiring additional services. Our results show that
basic serverless components---function schedulers and consistent data
stores---are sufficient abstractions for building complex and fault-tolerant
serverless applications. Moreover, \name{} affords applications more
flexibility, reduces costs and performs well compared with standalone
orchestrators with similar execution guarantees.