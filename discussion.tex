\section{Limitations \& Future Work}\label{sec:limitations}

\paragraph{Unsupported applications.} \name{} supports a superset of
applications that can be expressed using Step Functions, but there are
applications that do not fit \name{}'s constraints. In particular, \name{} only
supports statically defined control structures. For example, Durable Functions
expresses workflows dynamically as code and allows the developer to use
arbitrary logic to determine what the next workflow step should be at runtime.
This is not currently possible with \name{}.

\paragraph{Measurement error.} Due to the opaque design, implementation and
pricing of production workflow systems, such as Step Functions, comparisons in
our evaluations are limited in their explanatory power. In particular, we use
the current \emph{price} of Lambda, DynamoDB, and Step Functions as a proxy for
the \emph{cost} of providing these services. Of course, prices may be either
lower or higher for a particular service than the underlying cost.

\paragraph{Support for more platforms.} While \name{} is designed
to run on any serverless platform that meets our minimal criteria, our current
implementation is only complete for AWS Lambda using DynamoDB or S3 for storage.
We are working on other backends, including for Azure Functions and Google
Functions.

\section{Conclusion}\label{sec:conclusion}

Serverless platforms allow developers to construct applications from modular
programming units that can scale quickly and independently, promising
burst-scalability and fine-grained billing. Workflow orchestrators make building
complex applications out of these event-driven, asynchronous functions
reasonable, but are inflexible and may introduce performance, cost, scalability,
and deployment overhead to developers and platform providers. We designed and
implemented \name{}, a \emph{decentralized} workflow system that requires no
additional infrastructure to deploy, imposes no additional limits on
scalability, and performs as well as or better than centralized solutions while
providing similar expressiveness and execution guarantees.

Our results show that basic serverless building blocks---function schedulers and
consistent scalable datastores---are sufficient abstractions for building
complex, fault-tolerant, and scalable applications. Leveraging decentralized
orchestration, rather than centralized services can help performance, resource
usage, flexibility, and portability across cloud providers.
